%%%%%%%%%%%%%%%%%%%%%%%%%%%%%%%%%%%%%%%%%
% Professional Formal Letter
% LaTeX Template
% Version 2.0 (12/2/17)
%
% This template originates from:
% http://www.LaTeXTemplates.com
%
% Authors:
% Brian Moses
% Vel (vel@LaTeXTemplates.com)
%
% License:
% CC BY-NC-SA 3.0 (http://creativecommons.org/licenses/by-nc-sa/3.0/)
%
%%%%%%%%%%%%%%%%%%%%%%%%%%%%%%%%%%%%%%%%%

%----------------------------------------------------------------------------------------
%	PACKAGES AND OTHER DOCUMENT CONFIGURATIONS
%----------------------------------------------------------------------------------------

\documentclass[11pt, a4paper]{letter} % Set the font size (10pt, 11pt and 12pt) and paper size (letterpaper, a4paper, etc)

\input{structure1.tex} % Include the file that specifies the document structure % header settings part 1

%\longindentation=0pt % Un-commenting this line will push the closing "Sincerely," and date to the left of the page

%----------------------------------------------------------------------------------------
%	YOUR INFORMATION
%----------------------------------------------------------------------------------------

%\Who{Dr. Niko Bellic} % Your name

%\Title{, PhD} % Your title, leave blank for no title

\authordetails{
	\textbf{DEPARTEMEN FISIKA}\\ % Your department/institution
	Gedung F Fakultas Matematika dan Ilmu Pengetahuan Alam\\ % Your address
	Kampus UI Depok 16424\\ % Your city, zip code, country, etc
%	Email: j.smith@berkeley.edu\\ % Your email address
	Telp.  +62 21 78849008\\ % Your phone number
	\textbf{www.physics.ui.ac.id} % Your URL
}

%----------------------------------------------------------------------------------------
%	HEADER CONTENTS
%----------------------------------------------------------------------------------------

\logo{makara.jpg} % Logo filename, your logo should have square dimensions (i.e. roughly the same width and height), if it does not, you will need to adjust spacing within the HEADER STRUCTURE block in structure.tex (read the comments carefully!)

\headerlineone{UNIVERSITAS INDONESIA} % Top header line, leave blank if you only want the bottom line

\headerlinetwo{FAKULTAS MATEMATIKA DAN} % Bottom header line
\headerlinethree{ILMU PENGETAHUAN ALAM} % Bottom header line

%---------------------------------------------------------------------------------------- % header settings part 2

\renewcommand{\thefootnote}{\fnsymbol{footnote}}
\usepackage{xcolor}

\begin{document}

%----------------------------------------------------------------------------------------
%	TO ADDRESS
%----------------------------------------------------------------------------------------

\begin{letter}{}

%----------------------------------------------------------------------------------------
%	LETTER CONTENT
%----------------------------------------------------------------------------------------

\pagenumbering{arabic}
\opening{}

\noindent{\centering
\textbf{UJIAN TENGAH/AKHIR* SEMESTER TA 2019/2020\\
PROGRAM SARJANA FISIKA \\
MATA KULIAH (KODE MK)\\
TGL BLN THN \\
WAKTU: ......MENIT (00.00-00.00)\\}
}
\noindent
\begin{tabbing}
	\hspace{1.8cm}\=\kill
	Dosen: \> ...............(Kls A)\\
	\> ...............(Kls B)\\
	\> ...............(Kls C)
\end{tabbing} 

\noindent Sifat: Closed/open* book/note*. Boleh/tidak* menggunakan kalkulator. Boleh/tidak* menggunakan pensil.
\footnotetext{\textcolor{red}{* pilih salah satu}}
\noindent\hspace{0.\textwidth}\vhrulefill{1pt}\\

\noindent
Diberikan nilai konstanta sebagai berikut:\\
$g = 9,81$ m/s2\\
$N_A = 6,02x10^{23}$\\
$R = 8,31$ J/K.mol\\
\dots\dots

\begin{enumerate}
	\item Seorang penari iceskate bermassa 50 kg menari di atas lantai arena es pertunjukan. Pada saat badan dalam posisi tegak lurus lantai penari berputar dengan kecepatan putaran sebesar 180 rpm. Kemudian penari tersebut membungkukkan badan dan berpindah dari posisi awal dengan kecepatan 1 m/s. Bila momen inersia penari saat posisi tegak lantai 5 kg.m2 dan pada saat membungkuk 25 kg.m2, tentukan:
	
	\begin{enumerate}
		\item[a)] Besar kecepatan putaran saat penari membungkuk.
		\item[b)] Energi yang berubah bentuk saat penari akan bergerak pindah.
	\end{enumerate}

	\item Jfwoeirjwl
	\item gjwoleikwklkj
\end{enumerate}

Lorem ipsum dolor sit amet consectetur adipiscing, elit mi eros interdum vestibulum. Pretium primis gravida rutrum at ridiculus justo per aliquam sem vitae, congue libero feugiat ut lectus purus conubia ad curabitur, proin hac velit cum ultrices cras semper sed eget. Tristique magna fringilla phasellus orci pretium placerat nisi ultricies facilisis, justo ante enim nostra sodales rutrum primis convallis molestie nam, volutpat condimentum mauris netus lacus porttitor maecenas natoque.

Scelerisque tempor molestie nullam sociosqu metus nibh mus justo ante hendrerit, sapien mi curae ullamcorper rutrum parturient tristique netus viverra, lectus proin maecenas sociis a aliquam lobortis vestibulum platea. Dictumst justo a taciti et sociosqu, vestibulum vulputate luctus condimentum accumsan litora, turpis potenti suspendisse montes. Posuere curabitur eu sagittis feugiat aliquam integer viverra penatibus, facilisis nostra libero vehicula sapien laoreet pellentesque cras, etiam fermentum duis congue litora consequat hac.

Lorem ipsum dolor sit amet consectetur adipiscing, elit mi eros interdum vestibulum. Pretium primis gravida rutrum at ridiculus justo per aliquam sem vitae, congue libero feugiat ut lectus purus conubia ad curabitur, proin hac velit cum ultrices cras semper sed eget. Tristique magna fringilla phasellus orci pretium placerat nisi ultricies facilisis, justo ante enim nostra sodales rutrum primis convallis molestie nam, volutpat condimentum mauris netus lacus porttitor maecenas natoque.

Scelerisque tempor molestie nullam sociosqu metus nibh mus justo ante hendrerit, sapien mi curae ullamcorper rutrum parturient tristique netus viverra, lectus proin maecenas sociis a aliquam lobortis vestibulum platea. Dictumst justo a taciti et sociosqu, vestibulum vulputate luctus condimentum accumsan litora, turpis potenti suspendisse montes. Posuere curabitur eu sagittis feugiat aliquam integer viverra penatibus, facilisis nostra libero vehicula sapien laoreet pellentesque cras, etiam fermentum duis congue litora consequat hac.

\closing{}

%----------------------------------------------------------------------------------------
%	OPTIONAL FOOTNOTE
%----------------------------------------------------------------------------------------

% Uncomment the 4 lines below to print a footnote with custom text
%\def\thefootnote{}
%\def\footnoterule{\hrule}
%\footnotetext{\hspace*{\fill}{\footnotesize\em Footnote text}}
%\def\thefootnote{\arabic{footnote}}

%----------------------------------------------------------------------------------------

\end{letter}

\end{document}
