% ==========================================================================	%
%																			%
%           		 				Kata Pengantar								%
% 																			%
% ==========================================================================	%
%																			%
% INI PAKAI FORMAT 1 HALAMAN KATA PENGANTAR, JANGAN BERTELE-TELE				%
%																			%
% Pengaturan Halaman -------------------------------------------------------	%
%																			%
%Set nomor halaman															%
\setcounter{page}{5}															%
%Memasukkan ke daftar isi													%
\addcontentsline{toc}{chapter}{\numberline{}Kata Pengantar}					%
%																			%
% Isi ----------------------------------------------------------------------	%
%																			%
%Judul Halaman																%
\chapter*{KATA PENGANTAR}													%
%																			%
%Isi pengantar																%
Segala puji dan syukur kehadirat Allah SWT, yang telah melimpahkan rahmat, 
hidayah dan petunjuk-Nya kepada kita umat muslim. Shalawat dan salam semoga 
senantiasa tercurahkan pada Nabi Muhammad SAW, keluarga, para sahabat dan 
pengikutnya. 										
	
Bermula ketika masa SMA, penulis sangat tertarik dengan dunia fisika yang begitu banyak misteri. Kemudian lanjut ke dunia perkuliahan, dimana penulis diperkenalkan dengan cabang-cabang ilmu fisika yang tidak hanya terdiri dari mekanika klasik yang pernah diajarkan di SMA saja, tetapi juga ke bidang-bidang instrumentasi, geofisika, material, {\it condensed matter}, fisika medis dan biofisika, serta fisika nuklir dan partikel. Penulis kemudian masuk ke peminatan fisika nuklir dan partikel pada masa S1, kemudian melanjutkan studinya ke peminatan fisika medis dan biofisika, dimana dibidang tersebut penulis mengerjakan tugas akhirnya dalam bentuk tesis ini, yang memiliki tema kedokteran nuklir.
						
Dalam penulisan tesis ini, tak lepas dari bantuan dan dukungan berbagai 
pihak sehingga penulisan skripsi bisa selesai sesuai dengan apa yang penulis 
rencanakan. Oleh karena itu, patut kiranya penulis mengucapkan terimakasih 
kepada:
\begin{enumerate}
\item Allah SWT, yang telah memberikan penulis berbagai kemudahan, nikmat dan karunia yang tiada henti-hentinya kepada penulis. 
\item Ibu, kakak, serta kerabat keluarga yang sudah memberikan dukungan terhadap saya dalam hal penulisan skripsi ini.
\item Bapak Dr. Supriyanto A. Pawiro dan Ibu Prof. Dr. Drs. Djarwani S.S., selaku dosen pembimbing saya yang telah dengan sabar membimbing saya dengan sabar, memberikan wawasan pengetahuan sehingga penulis mampu menyelesaikan skripsi saya, dan memberikan nasihat-nasihat baik terkait dengan dunia perkuliahan dan non-perkuliahan.
%\item Dr. Anto Sulaksono, selaku dosen pembimbing akademik ketika saya masuk peminatan fisika nuklir dan partikel dan dosen penguji II saya yang telah yang telah bersedia memberikan banyak masukan, saran, motivasi, gambaran-gambaran mengenai dunia perkuliahan, dan bersedia berdiskusi dengan penulis selama penulisan skripsi sehingga penulis dapat mengikuti perkuliahan setiap semester dengan lancar tanpa ada suatu kendala yang berarti.
%\item Prof. Dr. Terry Mart, selaku dosen penguji I saya yang telah memberikan banyak masukan, saran, motivasi, dan bersedia berdiskusi dengan penulis selama penulisan skripsi.
\item Seluruh dosen Program Studi Magister Fisika FMIPA UI yang telah bersedia
membagi ilmu-ilmunya kepada penulis selama perkuliahan.
\item Teman-teman S2 fisika medis dan biofisika 2016, yang senantiasa memberikan dorongan kepada penulis, menjadi tempat berkeluh kesah, memberikan saran, tempat mengungkapkan ide-ide gila dan konyol. Bersama kalian, penulis masih dapat menikmati indahnya masa-masa perkuliahan di tengah materi yang padat dan rumit.
\item Teman-teman Lab Teori, terutama kepada Samson, Ilham, Reyhan, Nizar, Jason, dll, terima kasih atas dukungan dan doa sehingga penulis dapat menyelesaikan tesis ini tepat pada waktunya. Terima kasih atas persahabatan yang indah ini kawan.
\item Serta kepada semua pihak lain yang tidak dapat disebutkan satu persatu, yang telah memberikan dukungan kepada penulis baik dalam penyusunan tesis ini maupun dalam dunia perkuliahan. 
\end{enumerate}					

Akhir kata penulis berharap agar tesis ini bisa membawa manfaat bagi pengembangan ilmu pengetahuan di Indonesia. Penulis menyadari bahwa penulisan skripsi ini juga tidak luput dari kesalahan, untuk itulah penulis memohon maaf. Penulis juga memohon saran dan kritik untuk penyempurnaan. 
%
%Tempat & Tanggal Pengesahan													%     
\begin{flushright}															%
Depok, \tahun																%
\end{flushright}																%
\begin{flushright}															%
\penulis																		%
\end{flushright}																%
%																			%
% ++++++++++++++++++++++++++++++++++++++++++++++++++++++++++++++++++++++++++	%
%  							~~~~~ Selesai ~~~~~								%
% ++++++++++++++++++++++++++++++++++++++++++++++++++++++++++++++++++++++++++	%