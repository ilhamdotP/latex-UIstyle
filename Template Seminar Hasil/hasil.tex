% ==============================================================%
%																%
% 			TEMPLATE TESIS FISIKA MEDIS & BIOFISIKA				%
% 					Universitas Indonesia						%
%																%
%						  Developer:								%							
% 		   Aditya Tri Oktaviana (oktavianaadit@gmail.com)			%
%                       Copyright 2018							%
%																%
% ==============================================================%
% 						 PENDAHULUAN							%
% ==============================================================%
%																%
% TIDAK PERLU DIEDIT											% 
%																%
%Tipe Dokumen													%
\documentclass[a4paper,12pt]{uireport}							%
%																%
%Format Skripsi UI												%
\usepackage{uithesis}											%
%
\usepackage{pdfpages}											%
%
\usepackage{enumitem}
			%
%
%Memulai Dokumen												%
\begin{document}												%
%																%
%Identitas														%
% ======================================================================	%
%																		%
%                               IDENTITAS								%
% 																		%
% ======================================================================	%
%																		%
%ISI SESUAI IDENTITAS ASLI												%
%																		%
%Nama & NPM ------------------------------------------------------------	%
\def\penulis	{Aditya Tri Oktaviana}											%
\def\Penulis{ADITYA TRI OKTAVIANA}											%
\def\npm		{1606841130}													%
%																		%
%Institusi & Program studi----------------------------------------------	%
\def\departemen		{Departemen Fisika}									%
\def\dept			{Fisika}												%
\def\fakultas     	{Fakultas Matematika dan Ilmu Pengetahuan Alam}		%
\def\Fakultas		{FAKULTAS MATEMATIKA DAN ILMU PENGETAHUAN ALAM}		%
\def\peminatan 		{Fisika Medis & Biofisika}							%
\def\universitas  	{Universitas Indonesia}								%
\def\Universitas		{UNIVERSITAS INDONESIA}								%
\def\program			{S2 Reguler Ilmu Fisika}									%
\def\programeng		{Master Program in Physics}					%
\def\programstudi  	{Program Studi Ilmu Fisika}								%
%																		%
%Pembimbing & Penguji --------------------------------------------------	%
\def\pembimbingsatu 	{Dr. Supriyanto Ardjo Pawiro} 								%	
\def\pembimbingdua      {Prof. Dr. Djarwani S.S.}	 						%
\def\pengujisatu   	{Dr. Prawito} 							%
\def\pengujidua  	{Dr. I Putu Susila}										%
%																		%
% MahaKarya ------------------------------------------------------------	%
\def\judul  	{Simulasi Monte Carlo untuk Verifikasi Hasil Pengukuran \textit{Quality Control} Sistem SPECT dengan Phantom Jaszsczak}		%
\def\Judul		{Simulasi Monte Carlo untuk Verifikasi Hasil Pengukuran \textit{Quality Control} Sistem SPECT dengan Phantom Jaszsczak}		%
\def\title		{Monte Carlo Simulation for Verification Measurements Result of Quality Control SPECT System using Jaszsczak Phantom}	%
\def\karya		{Proposal Tesis}  												%
\def\Karya		{PROPOSAL TESIS}												%
\def\gelar		{sarjana} 												%
%																		%
% Tanggal Penting ------------------------------------------------------	%
\def\tahun		 	{November 2017}											%
\def\tanggallulus	{27 November 2017}										%
\def\tanggalsidang	{27 November 2017}										%
%																		%
%																		%
% ++++++++++++++++++++++++++++++++++++++++++++++++++++++++++++++++++++++	%
%																		%
%					      ~~~~~ Selesai ~~~~~							%
%																		%
% ++++++++++++++++++++++++++++++++++++++++++++++++++++++++++++++++++++++	%												%
%																%
%Ejaan untuk pemenggalan kata									%
% ==============================================================	%
%																%
%                EJAAAN YANG DISEMPURNAKAN (EYD)					%
% 																%
% ==============================================================	%
%																%
%SILAHKAN DITAMBAHKAN EJAAN KATA YANG BENAR						%
%																%
%Ejaan															%
\hyphenation{													%
	re-ak-si    													%
    in-va-ri-an   												%
}																%
%																%
%																%
% ++++++++++++++++++++++++++++++++++++++++++++++++++++++++++++++	%
%																%
% 					  ~~~~~ Selesai ~~~~~						%
%																%
% ++++++++++++++++++++++++++++++++++++++++++++++++++++++++++++++	%													%
%																%
%Halaman Depan													%
% ======================================================================	%
%																		%
%                               Halaman Depan							%
% 																		%
% ======================================================================	%
%																		%
% TIDAK PERLU DIEDIT														%
%																		%
% ======================================================================	%
%						Halaman pertama (cover depan)					%
% ======================================================================	%
\thispagestyle{empty} 													%
\begin{centering}														%
\includegraphics[width=2.5cm,height=2.5cm]{Gambar/makara.png} \\[0.4cm]	%
\textbf{																	%
	\universitas\\[3cm]													%
	\judul\\[3cm]														%
	\karya\\[3cm]														%
	\penulis\\															%
	\npm\\[3.5cm]														%
	\fakultas\\															%
	\programstudi\\														%
	Depok\\ 																%
	\tahun}\\															%
\end{centering}															%
%																		%
% ======================================================================	%
%                      Halaman kedua (cover dalam)						%
% ======================================================================	%
%																		%
%\newpage																	%
%\thispagestyle{empty}													%
%\begin{centering}														%
%\includegraphics[width=2.5cm,height=2.5cm]{gambar/makara.png}\\[0.4cm]	%
%\textbf{																	%
%	\universitas\\[3cm]													%
%	\judul\\[3cm]														%
%	\karya}\\															%
%Diajukan sebagai salah satu syarat untuk memperoleh gelar \gelar\\[2cm]	%
%\textbf{																	%
%	\penulis\\ 															%
%	\npm\\[4cm]															%
%	\fakultas\\															%
%	\programstudi\\ 														%
%	Depok\\ 																%
%	\tahun}\\															%
%\end{centering}															%
%\newpage																	%
%																		%
% ++++++++++++++++++++++++++++++++++++++++++++++++++++++++++++++++++++++	%
%                          ~~~~~ Selesai ~~~~~							%
% ++++++++++++++++++++++++++++++++++++++++++++++++++++++++++++++++++++++	%													%
%																%
%Tipe nomor halaman												%
\pagenumbering{roman}											%
%																%
%Pernyataan Orisinalitas										%
% ==========================================================================	%
%																			%
%           		        Halaman Pernyataan Orisinalitas						%
% 																			%
% ==========================================================================	%
%																			%
% TIDAK PERLU DIEDIT															%
%																			%
% Pengaturan Halaman -------------------------------------------------------	%
%																			%
%Set nomor halaman															%
\setcounter{page}{3}															%
%Menambahkan ke Daftar Isi													%
\addcontentsline{toc}{chapter}{\numberline{}Halaman Pernyataan Orisinalitas}	%
%																			%
% Isi ----------------------------------------------------------------------	%
%																			%
%Judul Halaman																%
\chapter*{HALAMAN PERNYATAAN ORISINALITAS}									%
\vspace{3cm}																	%
%																			%
%Pernyataan orisinalitas														%
\begin{centering}															%
\karya\; ini adalah hasil karya saya sendiri,\\								%
dan semua sumber baik yang dikutip maupun yang dirujuk\\						%
telah saya nyatakan dengan benar.\\[4.5cm]									%
%																			%
%Identitas																	%
\begin{tabbing}																%
\hspace{3.5cm}Nama	\hspace{2.5cm}	\= : \= \penulis \\[0.5cm]				%
\hspace{3.5cm}NPM 					\> : \> \npm \\[0.5cm]					%
\hspace{3.5cm}Tanda tangan			\> : \> \\[0.5cm]						%
\hspace{3.5cm}Tanggal 				\> : \> \tanggalsidang \\[0.5cm]			%
\end{tabbing}																%
\end{centering}																%
%																			%	
% ++++++++++++++++++++++++++++++++++++++++++++++++++++++++++++++++++++++++++	%
%                            ~~~~~ Selesai ~~~~~								%
% ++++++++++++++++++++++++++++++++++++++++++++++++++++++++++++++++++++++++++	%											%
%\includepdf{adit_ori.pdf}										%
%																%
%Halaman Pengesahan												%
% ======================================================================	%
%																		%
%           		 		    Halaman Pengesahan							%
% 																		%
% ======================================================================	%
%																		%
% TIDAK PERLU DIEDIT														%
%																		%
% Pengaturan Halaman ---------------------------------------------------	%
%																		%
%Set nomor halaman														%
\setcounter{page}{2}														%
%Memasukkan ke daftar isi												%
\addcontentsline{toc}{chapter}{\numberline{}Lembar Pengesahan}			%
%																		%
% Isi ------------------------------------------------------------------	%
%																		%
%Judul Halaman															%
\chapter*{HALAMAN PENGESAHAN}											%
%																		%
%Identitas																%
\begin{tabbing}															%
%Skripsi ini diajukan oleh\\												%
Nama		\hspace{1.5cm}	\= : \= \penulis\\								%
NPM 						\> : \> \npm\\									%
%Program Studi 			\> : \> \program\\								%
Judul Proposal 			\> : \> Simulasi Monte Carlo untuk Verifikasi Hasil Pengukuran \textit{Quality Control} \\
                        \>   \> Sistem SPECT dengan Phantom Jaszsczak \\ 								%
\end{tabbing}

%\begin{flushright}
%Depok, \hspace{0.1cm} November 2017	
%\end{flushright}													%
%																		%
%Pernyataan 																%
%\textbf{Telah berhasil dipertahankan di hadapan Dewan Penguji dan 
%	diterima sebagai bagian persyaratan yang diperlukan untuk memperoleh
%	gelar \gelar~ Sains pada Program Studi \program, \fakultas,
%	\universitas}														%
\vspace{0.2cm}															%
%																		%
%Pengesahan oleh Pembimbing & Penguji									%
\begin{center}															%
\bold{DEWAN PENGUJI}														%
\end{center}																%
\begin{tabular}{l l l l }												%
	& & & \\																%
	Pembimbing I&: & \pembimbingsatu & (\hspace*{3.0cm}) \\[1cm]			%
	& & & \\																%
	Pembimbing II&: & \pembimbingdua & (\hspace*{3.0cm}) \\[1cm]			%
	& & & \\																%
	Penguji &: &\pengujisatu & (\hspace*{3.0cm}) \\[1cm]				%
	& & & \\																%
	Penguji &: &\pengujidua & (\hspace*{3.0cm}) \\[1cm]				%
\end{tabular}															%
\vspace*{2cm}\\	

%\begin{tabular}{l l l l }												%
%	& & & \\																%
%	Pembimbing I&: & \pembimbingsatu & (\hspace*{3.0cm}) \\[1cm]			%
%	& & & \\																%
%	Pembimbing II&: & \pembimbingdua & (\hspace*{3.0cm}) \\[1cm]			%
%	& & & \\																%
%\end{tabular}															%
%\vspace*{2cm}\\															%
%																		%
%Tempat & Tanggal Pengesahan												%
%\begin{tabular}{l l l}													%
%	Ditetapkan di	&: & Depok\\											%
%	Tanggal			&: & \tanggallulus \\								%
%\end{tabular}															%
%																		%
% ++++++++++++++++++++++++++++++++++++++++++++++++++++++++++++++++++++++	%
% 							~~~~~ Selesai ~~~~~							%
% ++++++++++++++++++++++++++++++++++++++++++++++++++++++++++++++++++++++	%											%
%\includepdf{adit_pengesahan.pdf}								%
%
%Kata Pengantar													%
% ==========================================================================	%
%																			%
%           		 				Kata Pengantar								%
% 																			%
% ==========================================================================	%
%																			%
% INI PAKAI FORMAT 1 HALAMAN KATA PENGANTAR, JANGAN BERTELE-TELE				%
%																			%
% Pengaturan Halaman -------------------------------------------------------	%
%																			%
%Set nomor halaman															%
\setcounter{page}{5}															%
%Memasukkan ke daftar isi													%
\addcontentsline{toc}{chapter}{\numberline{}Kata Pengantar}					%
%																			%
% Isi ----------------------------------------------------------------------	%
%																			%
%Judul Halaman																%
\chapter*{KATA PENGANTAR}													%
%																			%
%Isi pengantar																%
Segala puji dan syukur kehadirat Allah SWT, yang telah melimpahkan rahmat, 
hidayah dan petunjuk-Nya kepada kita umat muslim. Shalawat dan salam semoga 
senantiasa tercurahkan pada Nabi Muhammad SAW, keluarga, para sahabat dan 
pengikutnya. 										
	
Bermula ketika masa SMA, penulis sangat tertarik dengan dunia fisika yang begitu banyak misteri. Kemudian lanjut ke dunia perkuliahan, dimana penulis diperkenalkan dengan cabang-cabang ilmu fisika yang tidak hanya terdiri dari mekanika klasik yang pernah diajarkan di SMA saja, tetapi juga ke bidang-bidang instrumentasi, geofisika, material, {\it condensed matter}, fisika medis dan biofisika, serta fisika nuklir dan partikel. Penulis kemudian masuk ke peminatan fisika nuklir dan partikel pada masa S1, kemudian melanjutkan studinya ke peminatan fisika medis dan biofisika, dimana dibidang tersebut penulis mengerjakan tugas akhirnya dalam bentuk tesis ini, yang memiliki tema kedokteran nuklir.
						
Dalam penulisan tesis ini, tak lepas dari bantuan dan dukungan berbagai 
pihak sehingga penulisan skripsi bisa selesai sesuai dengan apa yang penulis 
rencanakan. Oleh karena itu, patut kiranya penulis mengucapkan terimakasih 
kepada:
\begin{enumerate}
\item Allah SWT, yang telah memberikan penulis berbagai kemudahan, nikmat dan karunia yang tiada henti-hentinya kepada penulis. 
\item Ibu, kakak, serta kerabat keluarga yang sudah memberikan dukungan terhadap saya dalam hal penulisan skripsi ini.
\item Bapak Dr. Supriyanto A. Pawiro dan Ibu Prof. Dr. Drs. Djarwani S.S., selaku dosen pembimbing saya yang telah dengan sabar membimbing saya dengan sabar, memberikan wawasan pengetahuan sehingga penulis mampu menyelesaikan skripsi saya, dan memberikan nasihat-nasihat baik terkait dengan dunia perkuliahan dan non-perkuliahan.
%\item Dr. Anto Sulaksono, selaku dosen pembimbing akademik ketika saya masuk peminatan fisika nuklir dan partikel dan dosen penguji II saya yang telah yang telah bersedia memberikan banyak masukan, saran, motivasi, gambaran-gambaran mengenai dunia perkuliahan, dan bersedia berdiskusi dengan penulis selama penulisan skripsi sehingga penulis dapat mengikuti perkuliahan setiap semester dengan lancar tanpa ada suatu kendala yang berarti.
%\item Prof. Dr. Terry Mart, selaku dosen penguji I saya yang telah memberikan banyak masukan, saran, motivasi, dan bersedia berdiskusi dengan penulis selama penulisan skripsi.
\item Seluruh dosen Program Studi Magister Fisika FMIPA UI yang telah bersedia
membagi ilmu-ilmunya kepada penulis selama perkuliahan.
\item Teman-teman S2 fisika medis dan biofisika 2016, yang senantiasa memberikan dorongan kepada penulis, menjadi tempat berkeluh kesah, memberikan saran, tempat mengungkapkan ide-ide gila dan konyol. Bersama kalian, penulis masih dapat menikmati indahnya masa-masa perkuliahan di tengah materi yang padat dan rumit.
\item Teman-teman Lab Teori, terutama kepada Samson, Ilham, Reyhan, Nizar, Jason, dll, terima kasih atas dukungan dan doa sehingga penulis dapat menyelesaikan tesis ini tepat pada waktunya. Terima kasih atas persahabatan yang indah ini kawan.
\item Serta kepada semua pihak lain yang tidak dapat disebutkan satu persatu, yang telah memberikan dukungan kepada penulis baik dalam penyusunan tesis ini maupun dalam dunia perkuliahan. 
\end{enumerate}					

Akhir kata penulis berharap agar tesis ini bisa membawa manfaat bagi pengembangan ilmu pengetahuan di Indonesia. Penulis menyadari bahwa penulisan skripsi ini juga tidak luput dari kesalahan, untuk itulah penulis memohon maaf. Penulis juga memohon saran dan kritik untuk penyempurnaan. 
%
%Tempat & Tanggal Pengesahan													%     
\begin{flushright}															%
Depok, \tahun																%
\end{flushright}																%
\begin{flushright}															%
\penulis																		%
\end{flushright}																%
%																			%
% ++++++++++++++++++++++++++++++++++++++++++++++++++++++++++++++++++++++++++	%
%  							~~~~~ Selesai ~~~~~								%
% ++++++++++++++++++++++++++++++++++++++++++++++++++++++++++++++++++++++++++	%												%
%																%
%Halaman Persetujuan											%
% ==========================================================================	%
%																			%
%           		 	      Halaman Persetujuan Publikasi						%
% 																			%
% ==========================================================================	%
%																			%
% TIDAK PERLU DIEDIT 														%
%																			%
% Pengaturan Halaman -------------------------------------------------------	%
%																			%
%Set nomor halaman															%
\setcounter{page}{7}															%
%Memasukkan ke daftar isi													%
\addcontentsline{toc}{chapter}{\numberline{}Halaman Pernyataan
	Persetujuan Publikasi}													%
%																			%
% Isi ----------------------------------------------------------------------	%
%																			%
%Judul Halaman																%
\chapter*{
	PERNYATAAN PERSETUJUAN PUBLIKASI\\
	TUGAS AKHIR UNTUK KEPENTINGAN AKADEMIS}									%
\begin{center}																%
\vspace{-1.6cm}																%
\newcommand{\HRule}{\rule{\linewidth}{0.5mm}}								%
\HRule																		%
\end{center}																	%
%																			%	
%Pernyataan																	%
Sebagai sivitas akademik Universitas Indonesia, saya yang bertanda tangan 
di bawah ini;																%
%																			%
%Identitas																	%
\begin{tabbing}																%
Nama		\hspace{1.8cm}	\= : \= \penulis\\									%
NPM 						\> : \> \npm\\										%
Program Studi 			\> : \> \program\\									%
Departemen				\> : \> \dept\\										%
Peminatan 				\> : \> \peminatan\\									%
Fakultas 				\> : \> \fakultas\\									%
Jenis Karya 				\> : \> \karya\\										%
\end{tabbing}																%
%																			%
%Pernyataan2																	%
\noindent demi pengembangan ilmu pengetahuan, menyetujui untuk memberikan 
kepada Universitas Indonesia \textbf{Hak Bebas Royalti Noneksklusif 
(\textit{Non-exclusive Royalty Free Right})} atas karya ilmiah saya yang 
berjudul:																	%
%																			%
%Judul mahakarya																%
\begin{center}																%
\textbf{\judul}																%
\end{center}																	%
%																			%
%Pernyataan3																	%		
\noindent beserta perangkat yang ada (jika diperlukan). Dengan Hak Bebas 
Royalti Noneksklusif ini Universitas Indonesia berhak menyimpan, 
mengalihmedia/formatkan, mengelola dalam bentuk pangkalan data 
(\textit{database}), merawat, dan memublikasikan tugas akhir saya selama 
tetap mencantumkan nama saya sebagai penulis/pencipta dan sebagai pemilik 
Hak Cipta.\\[0.2cm]															%
Demikian Pernyataan ini saya buat dengan sebenarnya.							%
%																			%
%Tempat & Tanggal Pengesahan													%
\begin{flushright}															%
Depok, \tanggallulus \\[1cm]													%
\end{flushright}																%
\begin{flushright}															%
\penulis																		%
\end{flushright}																%
%																			%
% ++++++++++++++++++++++++++++++++++++++++++++++++++++++++++++++++++++++++++	%
% 							 ~~~~~ Selesai ~~~~~								%
% ++++++++++++++++++++++++++++++++++++++++++++++++++++++++++++++++++++++++++	%											%
%\includepdf{adit_pernyataan}									%
%																%
%Abstrak														%
% ==========================================================================	%
%																			%
%           		 					Abstrak									%
% 																			%
% ==========================================================================	%
%																			%
% SANGAT PENTING UNTUK DIEDIT												%
%																			%
% Pengaturan Halaman -------------------------------------------------------	%
%																			%
%Set nomor halaman															%
\setcounter{page}{8}															%
%Memasukkan ke daftar isi													%
\addcontentsline{toc}{chapter}{\numberline{}Abstrak}							%
%																			%
% ==========================================================================	%
% 							  ABSTRAK (INDONESIA)							%
% ==========================================================================	%
%																			%
%Judul Abstrak																%
\chapter*{ABSTRAK}															%
%																			%
%Identitas																	%
\begin{tabbing}																%
Nama		\hspace{1.5cm}	\= : \= \penulis\\									%
Program Studi 			\> : \> \program\\									%
Judul 		 			\> : \> Simulasi Monte Carlo untuk Verifikasi Hasil Pengukuran \\
                        \>   \> \textit{Quality Control} Sistem SPECT dengan Phantom Jaszsczak \\										%
\end{tabbing}																%
%																			%
%Isi Abstrak																	%
\noindent Citra kualitas tinggi diperlukan dalam memperoleh informasi yang akurat dalam menentukan posisi anatomi atau fungsional organ tubuh. Pada kedokteran nuklir, salah satu sistem pencitraan yang digunakan adalah SPECT (\textit{Single Photon Emission Computed Tomography}). Untuk menghasilkan citra kualitas tinggi diperlukan inspeksi dan validasi kuantitatif dari pengukuran sistem SPECT, sehingga QC (\textit{Quality Control}) sistem SPECT harus dilakukan untuk melihat performa dari pencitraan tersebut. Komponen seperti uniformitas, kontras dan resolusi spasial menjadi variabel penting dalam melihat hasil pengukuran dari sistem SPECT yang telah dilakukan. Dalam pengukuran, artefak dapat terjadi sehingga menurunkan kualitas citra yang dihasilkan.  Pada penelitian ini dilakukan ketiga pengukuran tersebut secara bersamaan dengan menggunakan fantom Jaszsczak. Pengukuran dilakukan secara manual berdasarkan protokol standar QC, dan sebagai justifikasi pengukuran manual akan dilakukan simulasi Monte Carlo sebagai data pembanding.
\vspace*{0.6cm}\\															%
%																			%
%Kata Kunci																	%
\noindent \textbf{Kata kunci:}\\												%
Citra, Fantom Jaszsczak, \textit{Quality Control}, Simulasi Monte Carlo, SPECT 																	%
%																			%
% ==========================================================================	%
% 								ABSTRAK (ENGLISH)							%
% ==========================================================================	%
%																			%
%Judul Abstrak																%
\chapter*{ABSTRACT}															%
%																			%
%Identitas																	%
\begin{tabbing}																%
Name		\hspace{2.2cm}	\= : \= \penulis\\									%
Program of Study 		\> : \> \programeng\\								%
Title 					\> : \> Monte Carlo Simulation for Verification Measurements  \\ 
						\>   \> Result of Quality Control SPECT System using Jaszsczak \\
						\>   \> Phantom \\										%
\end{tabbing}																%
%																			%
%Isi Abstrak																	%
\noindent High quality image is required in obtaining accurate information in determining the anatomical or functional position of organs. In nuclear medicine, one of the imaging systems used is SPECT (Single Photon Emission Computed Tomography). To produce high quality imagery required quantitative inspection and validation of SPECT system measurements, so SPECT system quality control (QC) should be performed to see the performance of the imaging. Components such as spatial resolution, contrast, and inhomogeneity are important variables in viewing the measurement results of SPECT systems that have been done. In the measurement, artifacts can occur thus decreasing the quality of the resulting image. In this study, these three measurements were performed simultaneously using Jaszsczak phantom. Measurements are done manually based on standard QC protocols, and as a justification for manual measurements Monte Carlo simulations will be performed as comparison data.
\vspace*{0.6cm}\\															%
%																			%
%Kata Kunci																	%
\noindent \textbf{Keywords:}\\												%
Image, Jaszczak Phantom, Quality Control, Monte Carlo Simulation, SPECT		%
%																			%
% ++++++++++++++++++++++++++++++++++++++++++++++++++++++++++++++++++++++++++	%
% 							~~~~~ Selesai ~~~~~								%
% ++++++++++++++++++++++++++++++++++++++++++++++++++++++++++++++++++++++++++	%												%
%																%
% ==============================================================%
%						 DAFTAR ISI								%
% ==============================================================%
%																%
%Membuat Halaman daftar isi										%
\clearpage														%
%Set Halaman daftar isi											%
\setcounter{page}{10}											%
%Memasukkan ke daftar isi										%
\addcontentsline{toc}{chapter}{\numberline{}Daftar Isi}			%
%Daftar isi														%
\tableofcontents												%
%																%
%Daftar Gambar													%
\listoffigures													%
%Memasukkan ke daftar isi										%
\addcontentsline{toc}{chapter}{\numberline{}Daftar Gambar} 		%
%																%
%Daftar tabel													%
\listoftables													%
%Memasukkan ke daftar isi										%
\addcontentsline{toc}{chapter}{\numberline{}Daftar Tabel} 		%
\clearpage														%
%																%
% ==============================================================%
%					    ISI MAHAKARYA							%
% ==============================================================%
%																%
%Tipe nomor halaman												%
\pagenumbering{arabic}											%
%																%
%BAB 1 - PENDAHULUAN											%
% +++++++++++++++++++++++++++
%
% BAB 1: PENDAHULUAN
%
% +++++++++++++++++++++++++++

\chapter{PENDAHULUAN}

\section{Latar Belakang}
Citra dalam medis merupakan salah satu faktor untuk melakukan perlakuan terapi medis lebih lanjut baik dalam radioterapi maupun kedokteran nuklir. Dalam hal ini kualitas citra yang dihasilkan merupakan informasi yang sangat penting dalam menunjukkan posisi anatomi ataupun fungsional organ tubuh, sehingga diperoleh diagnosa yang cukup akurat. Untuk menghasilkan citra yang baik diperlukan modalitas dengan ketelitian tinggi sehingga optimalisasi kuantitas yang dihasilkan, baik resolusi spasial, sensitivitas, maupun kontras, diperoleh dengan baik. Modalitas pencitraan yang digunakan terdiri dari dua jenis, yaitu modalitas dalam memberikan informasi detail mengenai anatomi pasien (MRI, CT-Scan, X-Rays), dan modalitas dalam memberikan informasi mengenai aktivitas fungsional dari suatu organ atau jaringan spesifik (SPECT, PET).\cite{Hirtl}

SPECT (\textit{Single Photon Emission Computed Tomography}) dan PET (\textit{Positron Emission Tomography}) merupakan modalitas pencitraan diagnostik dalam kedokteran nuklir. Kedua modalitas tersebut menggunakan metode non-invasif, yaitu menggunakan injeksi radioaktif kedalam tubuh pasien dimana radioaktif dalam tubuh tersebut akan dievakuasi oleh organ tubuh sehingga mampu dideteksi oleh detektor. Dalam merekonstuksi citra keduanya menggunakan metode \textit{filtered backprojection} atau metode iteratif. Perbedaan dari kedua modalitas tersebut jika SPECT memanfaatkan distribusi radionuklida pemancar sinar-X atau sinar gamma (contoh \textsuperscript{99m}Tc) pada pasien, sedangkan PET memanfaatkan distribusi radionuklida pemancar positron (contoh \textsuperscript{18}FDG). PET memiliki keunggulan dalam hal sensitivitas daripada SPECT, namun kelemahan PET adalah mengeluarkan biaya yang cukup tinggi.\cite{Ferreira}

Seperti yang telah disebutkan sebelumnya bahwa pencitraan dengan menggunakan SPECT ataupun PET menggunakan material radioaktif yang dimasukkan kedalam tubuh. Material radioaktif yang digunakan disebut sebagai radiofarmaka yang memiliki sifat inti atom tidak stabil. Beberapa radiofarmaka yang digunakan diantaranya adalah \textsuperscript{123}I, \textsuperscript{111}In, \textsuperscript{57}Co, \textsuperscript{201}Th, \textsuperscript{99m}Tc, dsb. Pemilihan radiofarmaka untuk pencitraan dapat dilihat dari sifat pemancar gamma isotop dari beberapa jenis radioaktif tersebut. Misal \textsuperscript{111}In, \textsuperscript{131}I, \textsuperscript{125}I, dan \textsuperscript{99m}Tc merupakan jenis radioaktif pemancar gamma isotop sehingga jenis tersebut sangat cocok digunakan sebagai material radioaktif. Perlu diperhatikan faktor lain seperti waktu peluruhan dan nilai distribusi dosis dalam memilih radiofarmaka yang sesuai untuk pencitraan, mengingat proteksi radiasi untuk keselamatan pasien dan pekerja sangat penting.\cite{Alehyani}\cite{Roshan}

Penggunaan SPECT dan PET sebagai pencitraan diagnostik pada klinis tidak lepas dari performa alat sistem untuk memperoleh citra kualitas tinggi, sehingga diperlukan protokol akuisisi terstandarisasi untuk QC (\textit{Quality Control}). Kualitas citra tinggi diperoleh dari banyaknya data cacahan, waktu panjang akuisisi, dan dosis radionuklida dalam jumlah besar. Namun pada klinis waktu akuisisi pendek dan dosis radionukliad rendah diutamakan mengingat keselamatan pasien dan staf. Ketidaktepatan dalam melakukan protokol QC terstandarisasi akan menghasilkan citra kualitas rendah yang mengakibatkan kesalahan deteksi lesion dan diagnosis.\cite{D'Arienzo1}

Perlu diperhatikan bahwa sensitivitas yang tinggi akan mengakibatkan resolusi rendah, dan begitu juga sebaliknya. Dalam pencitraan, resolusi spasial merupakan faktor yang sangat penting dalam memberikan kualitas citra sehingga sensitivitas biasanya dikorbankan. Kehilangan cacah dalam keseluruhan sistem akan menurunkan kualitas citra sehingga perlu dilakukan optimasi dengan mengontrol jumlah aktivitas yang diberikan dan waktu pencitraan yang realistis.\cite{Zhou} 

Komponen penting dalam pembentukan citra adalah proses scanning dengan menggunakan kamera gamma yang mampu mengoleksi data dari pasien secara simultan pada area yang diinginkan. Detektor yang digunakan adalah detektor sintilasi seperti kristal NaI(Tl) yang dilengkapi kolimator untuk memperoleh hubungan spasial antara titik pemancar radioaktif dalam tubuh dengan titik interaksi dengan kristal.\cite{Wolf}

Bentuk dan dimensi kolimator menjadi parameter geometrikal utama yang perlu diperhatikan. Ukuran kristal menentukan resolusi spasial. Pada beberapa kasus ukuran piksel dan lubang kolimator tidak sama, sehingga mengakibatkan sulitnya menemukan posisi relatif piksel dan lubang kolimator. Hal tersebut berdampak pada penurunan resolusi.\cite{Loudos1}

Artefak merupakan salah satu faktor menurunkan kualitas citra. Faktor koreksi hamburan, koreksi atenuasi, dan pergerakan pencitraan secara coincidence akan mengakibatkan munculnya artefak. Selain hal tersebut, dalam rekonstruksi citra akan terjadi artefak pula ketika algoritma yang digunakan salah.

Pada penelitian ini berfokus pada pengukuran menggunakan SPECT sebagai modalitas pencitraan. Mengingat QC dari SPECT perlu dilakukan dalam hal validasi kuantitaif seperti resolusi spasial, kontras, dan inhomogenitas. Phantom Jaszczak (Jenis \textit{Deluxe Flangeless}) digunakan dalam melakukan pengukuran ini karena dapat menginvestigasi tiga properti sistem tersebut dalam satu pengukuran secara bersamaan. Sebagai justifikasi dari pengukuran akan dilakukan simulasi Monte Carlo dimana hasil dari simulasi tersebut akan dibandingkan dengan hasil pengukuran. Diharapkan dalam hasil penelitian ini mampu menghasilkan sebuah protokol yang dapat digunakan untuk keperluan medis dan hasil analisis mampu memberikan interpretasi klinis yang akan memudahkan dalam perlakuan medis selanjutnya.

\section{Rumusan masalah}
Walaupun pengukuran sistem SPECT dengan phantom Jaszczak jenis \textit{Deluxe Flangeless} sudah memiliki protokol yang terstandarisasi, namun pada kenyataannya pengukuran tersebut hasilnya bervariatif (bergantung terhadap pengalaman dari pengukur) sehingga agak sulit menentukan protokol mana yang lebih baik digunakan. Pengukuran yang dilakukan dengan dua cara; manual dan simulasi, akan dianalisis dengan menbandingkan hasilnya sehingga diharapkan hasil tersebut menjadi acuan bagi tiap rumah sakit dalam membuat protokol QC terstandarisasi. Diharapkan dari hasil tersebut mampu dipelajari pendeteksian artefak yang terjadi pada pengukuran tersebut.

\section{Tujuan Penelitian}
Tujuan dari penelitian ini adalah sebagai berikut:

\begin{enumerate}[label=\alph*]
\item Melakukan evaluasi terhadap QC dari SPECT dan verifikasi distribusi dosis untuk memperoleh citra kualitas tinggi dan proteksi radiasi.
\item Mendeteksi artefak sehingga dapat meminimalisirnya.
\item Algoritma yang dihasilkan dapat digunakan untuk pembuatan software analisis pengukuran SPECT sehingga menghasilkan protokol untuk beberapa rumah sakit.
\end{enumerate}

\section{Manfaat Penelitian}
Penelitian ini bermanfaat sebagai pertimbangan dalam melakukan verifikasi QC sistem SPECT yang ada di rumah sakit Indonesia, serta validasi dari pengukuran sehingga menghasilkan protokol dan software yang dapat digunakan di rumah sakit.

\section{Batasan Penelitian}
Penelitian ini menggunakan SPECT/CT varian Siemens yang nantinya hanya menggunakan pengukuran SPECT dengan multihead gamma kamera tanpa menggunakan sistem CT. Pengukuran dengan menggunakan phantom Jazsczak jenis \textit{Flangeless Deluxe} untuk memperoleh resolusi spasial, kontras, dan inhomogenitas. Jenis radionuklida yang digunakan adalah \textsuperscript{99m}Tc. Untuk melakukan simulasi Monte Carlo menggunakan code GATE.

\section{Sistematika Penulisan}
Tesis yang akan dibuat setelah proposal terdiri dari sistematika penulisan yang terbagi menjadi beberapa sub-bab sebagai berikut:
\begin{enumerate}[label=\alph*]
\item BAB 1 PENDAHULUAN \\
Bab ini memuat latar belakang, rumusan masalah, tujuan penelitian, manfaat penelitian, batasan penelitian, dan sistematika penelitian.
\item BAB 2 TINJAUAN PUSTAKA \\
Bab ini memuat landasan teori yang berlandaskan dari penelitian yang akan dilakukan.
\item BAB 3 METODOLOGI PENELITIAN \\
Bab ini membahas tentang alat dan bahan apa saja yang akan digunakan dalam penelitian serta metode penelitian yang digunakan dalam melakukan proses pengambilan data.
\item BAB 4 HASIL DAN PEMBAHASAN \\
Bab ini memuat tentang hasil data-data penelitian yang telah dilakukan dan membahas hasil tersebut.
\item BAB 5 KESIMPULAN DAN SARAN \\
Bab ini memuat tentang kesimpulan yang diambil berdasarkan hasil dari data penelitian yang telah dilakukan serta saran dan diskusi yang ditambahkan untuk penelitian lebih lanjut.
\end{enumerate}													%
%																%
%BAB 2 - TEORI DASAR											%
% +++++++++++++++++++++++++++
%
% BAB 2: TEORI DASAR
%
% +++++++++++++++++++++++++++

\chapter{TINJAUAN PUSTAKA}

\section{SPECT}
SPECT pertama kali ditemukan oleh David Kuhl dan rekannya awal tahun 1960-an, atau sekitar 10 tahun sebelum ditemukannya CT X-ray oleh Hounsfield. Walaupun demikian penggunaan SPECT dalam klinis baru dimulai pada thaun 1980-an. Berbeda dengn CT, citra SPECT dapat ditampilkan dalam citra planar kedokteran nuklir.

Sistem SPECT biasanya terdiri dari beberapa kepala kamera sintilasi. Kamera tersebut mengakuisisi dan membentu citra dari distribusi radionuklida pemancar sinar-X atau gamma yang dihasilkan dari tubuh pasien. Proyeksi citra berupa data piksel umumnya diperoleh dari 180° untuk jantung dan 360° untuk yang bukan jantung. Rekonstruksi citra yang dibentuk dilakukan dengan menggerakkan kepala kamera sintilasi dari beberapa sudut dan dapat dilakukan dengan dua cara, yaitu secara kontinu selama kepala kamera bergerak, ataupun pada saat kepala kamera berhenti disudut tertentu (disebut dengan \textit{step and shoot aquisition}).

Untuk memperoleh resolusi spasial tinggi pada sistem SPECT menggunakan kolimator. Pada umumnya kolimator yang digunakan adalaha kolimator jenis \textit{parallel-hole}. Seiring berkembangnya modalitas pencitraan diciptakan kolimator khusus seperti kolimator \textit{fan-beam}, yang merupakan kolimator hibrid dari kolimator konvergen dan kolimator parallel. Kolimator tersebut merupakan jenis kolimator \textit{parallel-hole} arah-y, sehingga tiap baris piksel pada proyeksi citra berkaitan dengan satu transaksial slice dari subjek. Untuk arah-x, kolimator tersebut merupakan kolimator konvergen dengan karakteristik resolusi spasial dan efisiensi lebih tinggi daripada kolimator \textit{parallel-hole}.\cite{Bushberg}

Pencitraan dengan SPECT memiliki keterbatasan fundamental yang sama dengan pencitraan tomografi lainnya,diantaranya:

\begin{enumerate}[label=\alph*]
\item \textit{Collection efficiency}, yaitu berdasarkan radiasi gamma yang dipancarkan ke segala arah lapisan. Walaupun demikian untuk pencitraan yang masuk hanya ke detektor saja sehingga efisiensi sangat terbatas. Hal tersebut dapat diatasi dengan pasien dikelilingi oleh detektor.
\item Atenuasi radiasi gamma oleh pasien akibat penyederhanaan dengan melakukan pencacahan pada dua detektor yang saling berhadapan ataupun beberapa detektor, sehingga dibutuhkan faktor koreksi. Meski demikian koreksi ketelitian atenuasi tidak diperlukan dalam SPECT.
\item Masalah umum pada kedokteran nuklir dalam waktu pengoleksian hanya fraksi waktu radiasi gamma yang dipancarkan sehingga citra yang dibentuk dengan foton sangatlah terbatas.
\end{enumerate}

\section{Radioaktivitas}
Sifat unsur radioaktif yang digunakan pada dasarnya merupakan proses rambang sehingga sulit dalam memprediksi peluruhan atom. Tetapi laju transformasi dari peluruhan atom radioaktif tersebut dapat diamati dalam waktu lama. Dengan demikian jumlah atom yang meluruh per satuan waktu akan sebanding dengan jumlah atom radioaktif dalam sumber, atau didefinisikan dengan:

\begin{eqnarray}
\frac{dN}{dt} \sim N \label{eq1}
\end{eqnarray}

Dari persamaan (\ref{eq1}) dapat dinyatakan konstanta peluruhan ($\lambda$):

\begin{eqnarray}
-\frac{dN}{dt} = \lambda N \label{eq2}
\end{eqnarray}

Hubungan dengan aktivitas (\textit{A}) dimana \textit{A} didefinisikan sebagai perubahan jumlah atom per satuan waktu dengan $\lambda$ adalah:

\begin{eqnarray}
A = \lambda N \label{eq3}
\end{eqnarray}

Dari persamaan (\ref{eq3}) dapat ditentukan waktu paruh ($T_{1/2}$) berdasarkan pengamatan peluruhan sehingga:

\begin{eqnarray}
T_{1/2} = \frac{\ln 2}{\lambda} = \frac{0.693}{\lambda} \label{eq4}
\end{eqnarray}

Pada dasarnya inti atom terdiri dari beberapa nukleon yang dikenal dengan proton dan neutron. Kedua nukleon tersebut diikat dengan gaya nuklir kuat yang mengakibatkan nukleon tersebut tidak keluar dari inti atom. Pada inti atom yang radioaktif, nukleon akan memiliki energi yag cukup untuk keluar dari inti atom dan menghasilkan inti atom baru. Inti atom radioaktif kemungkinan disebabkan memiliki sifat berikut:

\begin{enumerate}[label=\alph*]
\item Kelebihan proton dan neutron sehingga memungkinkan inti atom secara spontan memancarkan partikel $\alpha$ (dengan 2 proton dan 2 neutron) agar menjadi inti atom stabil.
\item Kelebihan proton yang mengakibatkan inti atom memancarkan partikel $\beta^+$ dan melakukan penangkapan elektron dalam proses menjadi inti atom stabil.
\item Kelebihan neutron yang mengakibatkan inti atom memancarkan partikel $\beta^-$ dan kemungkinan akan diikuti oleh proses konversi internal.
\item Kelebihan energi akibat memancarkan partikel $\alpha$, $\beta^+$, ataupun $\beta^-$ sehingga untuk proses menuju inti atom stabil akan memancarkan $\gamma$.
\end{enumerate}

Dalam proses peluruhan menuju inti atom stabil berlaku beberapa hukum kekekalan, yaitu; kekekalan energi, kekekalan momentum linear, kekekalan momentum sudut, kekekalan muatan, kekekalan nomor massa.\cite{Bushberg}

\section{Performa}
Pengukuran kinerja kamera sintilasi merupakan salah satu indikator terbaik dalam menunjukkan kinerja klinis dari sistem pencitraan kedokteran nuklir. Baik untuk sistem SPECT maupun PET terbagi atas dua jenis pengukuran, yaitu pengukuran intrinsik dan pengukuran ekstrinsik. Pengukuran intrinsik adalah pengukuran kinerja kamera sintilasi tanpa menggunakan kolimator, sedangkan pengukuran ekstrinsik adalah pengukuran yang menggunakan kolimator. Umumnya pengukuran intrinsik lebih bermanfaat untuk membandingkan kinerja antar unit karena terpisah dari kolimator.

Uniformitas dapat digunakan sebagai acuan dalam menunjukkan tanggapan terhadap radiasi uniform pada permukaan detektor. Uniformitas dengan tanggapan ideal akan memberikan hasil citra yang uniform. Pada umumnya uniformitas intrinsik diukur dengan meletakkan sumber radionuklida di depan detektor tanpa kolimator. Sistem kolimator dan defek kamera dapat dievaluasi dengan meletakkan sumber radionuklida dipermukaan kamera. Hasil dari uniformitas citra dapat dianalisis secara manual ataupun dengan komputer.

Resolusi spasial merupakan ukuran kemampuan kamera dalam mencitrakan variasi spasial konsentrasi aktivitas dan membedakan obyek radioaktif yang berdekatan. Resolusi spasial dievaluasi dengan mengambil citra sumber garis untuk menentukan garis \textit{line spread function} (LSF). Kemudian ditentukan pula nilai \textit{full width at half maximum} (FWHM) dan \textit{full width at tenth maximum} serta \textit{modulation transfer function} (MTF). Resolusi spasial dari suatu sistem ($R_s$) dapat ditentukan oleh resolusi kolimator ($R_c$) dan resolusi intrinsik kamera ($R_i$) dengan mengikuti hubungan:

\begin{eqnarray}
R_s = \sqrt{{R_c}^2 - {R_i}^2} \label{eq5}
\end{eqnarray}

Beberapa jenis kolimator memiliki karakteristik dalam memperbesar dan memperkecil citra. Untuk jenis kolimator tersebut resolusi sistem dikoreksi dengan ukuran citra pada kristal atau objek yang sering disebut sebagai magnifikasi (\textit{m}), sehingga mengikuti persamaan:

\begin{eqnarray}
R_s = \sqrt{{R'_c}^2 - ({R_i}/{m})^2} \label{eq6}
\end{eqnarray}

Nilai $R'_s = R_s / m$ dan $R'_c = R_c / m$. Nilai \textit{m} ditentukan sebagai berikut:

\begin{itemize}
\item $m = 1.0$ untuk kolimator parallel
\item $m = f / (f - x)$ untuk kolimator konvergen
\item $m = f / (f + x)$ untuk kolimator divergen
\end{itemize}

\textit{f} adalah jarak dari kristal ke titik fokal kolimator, dan \textit{x} adalah jarak obyek dari kristal. Dari persamaan (\ref{eq6}) dapat dilihat bahwa magnifikasi kolimator menurunkan efek gangguan resolusi spasial intrinsik pada keseluruhan resolusi sistem.

Linearitas spasial adalah ukuran kemampuan kamera dalam mencitrakan bentuk obyek secara teliti. Linearitas spasial dievaluasi dengan melihat citra phantom bar atau melihat citra phantom lain dari nilai kelurusan garis dalam citra.

\textit{Multienergy spatial registration} atau disebut juga sebagai \textit{multiple window spatial registration} merupakan ukuran kemampuan kamera dalam mempertahankan magnifikasi citra yang sama, dan tidak tergantung pada deposit energi oleh sinar-X atau radiasi gamma dalam kristal. Foton dengan energi tinggi akan menghasilkan sinyal lebih besar dibandingkan dengan foton energi rendah. Pencitraan posisi objek ditentukan oleh normalisasi sinyal posisi X dan Y pada energi sinyal. Untuk radionuklida yang memancarkan beberapa energi foton, maka hasil citra merupakan superposisi dari beberapa citra berdasarkan energi foton tersebut yang berbeda magnifikasi.

Efisiensi sistem dari kamera sintilasi merupakan fraksi radiasi gamma yang dipancarkan oleh suatu sumber yang menghasilkan cacahan pembentuk citra. Efisiensi ini sangat penting karena berhubungan dengan waktu pencacahan dan menentukan jumlah \textit{quantum  mottle} dalam citra. Efisiensi ($E_s$) dipengaruhi oleh tiga faktor, yaitu:

\begin{enumerate}[label=\alph*]
\item Efisiensi kolimator ($E_c$), yaitu fraksi foton yang dipancarkan sumber ketika menembus lubang kolimator.
\item Efisiensi intrinsik kristal ($E_i$), yaitu fraksi foton yang menembus kolimator dan berinteraksi dengan kristal NaI(Tl). Efisiensi ini ditentukan oleh ketebalan kristal dan energi foton:

\begin{eqnarray}
E_i = 1 - e^{\mu x} \label{eq7}
\end{eqnarray}

Dengan $\mu$ adalah koefisien atenuasi kristal dan x adalah ketebalan kristal.
\item Fraksi foton (\textit{f}) yang sampai dan berinteraksi dengan kristal pada rangkaian diskriminasi.
\end{enumerate}

Efisiensi sistem dapat dituliskan dalam persamaan:

\begin{eqnarray}
E_s=E_c\times E_i\times f \label{eq8}
\end{eqnarray}

Pada umumnya kolimator parallel energi rendah untuk semua kegiatan memiliki nilai sekitar $2 \times 10^{-4}$, dan untuk energi rendah dengan efisiensi tinggi mempunyai nilai efisiensi sekitar $1 \times 10^{-4}$.

Resolusi energi suatu kamera sintilasi adalah ukuran kemampuan dalam membedakan antara interaksi deposisi berbagai energi yang berbeda dalam kristal. Kamera dengan resolusi energi superior mampu menghilangkan sebagian besar foton yang dihamburkan dalam tubuh pasien, atau foton tersebut telah mengalami interaksi \textit{coincidence}, sehingga menghasilkan citra dengan kontras yang lebih baik dan noise rambang yang relatif rendah. Resolusi energi diukur dengan menggunakan kamera dan mengekspos sumber titik radionuklida monoenergi gamma untuk memperoleh spektrum energi. Kemudian resolusi energi diperoleh dengan evaluasi FWHM \textit{photopeak spectrum} dibagi oleh energi foton dan dinyatakan dalam persentase.

Untuk performa laju cacah kamera pada umumnya ditentukan oleh laju cacah yang diamati pada kehilangan cacah 20\% dan laju cacah maksimum. Kedua laju cacah tersebut diukur dengan atau tanpa koreksi hamburan. Jika faktor hamburan diperhitungkan, maka kedua laju cacah tersebut akan menurun bila diukur. Pada umumnya laju cacah tinggi dicapai dengan menurunkan resolusi spasial dan resolusi energi.\cite{Bushberg}

\section{Artefak}
Sistem pencitraan tomografi SPECT dan kamera gamma memiliki mode operasi yang kompleks. Dalam pencitraan bertambahnya koreksi hamburan, koreksi atenuasi, pergerakan pencitraan secara \textit{coincidence} ketika melakukan uji alat atau klinis mengakibatkan munculnya artefak pada citra. Artefak dapat muncul akibat dari pengaruh radiofarmaka yang digunakan, operasi kamera gamma, pasien, atau prosedur sistem komputer seperti rekonstruksi citra. Salah satu artefak yang umumnya muncul berasal dari uniformitas yang menghasilkan ring artefak pada citra. 

Salah satu contoh munculnya artefak pada kamera gamma diakibatkan:

\begin{enumerate}
\item PMT (\textit{photomultiplier tube}) tidak memberikan respon sinyal yang ditangkap akibat adanya kerusakan.
\item Pergerakan detektor.
\item Reduksi aktivitas pada dinding septa.
\item Kesalahan penentuan \textit{Center of Rotation} (COR).
\end{enumerate}

\begin{figure}[h]
\includegraphics[scale=0.75]{Gambar/artifact1.png}
\caption{Contoh hasil citra tanpa dan dengan \textit{ring} artefak akibat reduksi aktivitas pada dinding septa\cite{O'Connor1}}
\label{artifact1}
\end{figure}

\begin{figure}[h]
\includegraphics[scale=0.75]{Gambar/artifact2.png}
\caption{10 citra diperoleh dengan perubahan sistem uniformitas selama 360$^\circ$ dimana artefak muncul antara 180$^\circ$ dan 252$^\circ$ akibat kelainan salah satu PMT\cite{O'Connor1}}
\label{artifact2}
\end{figure}

\begin{figure}[h]
\includegraphics[scale=0.75]{Gambar/artifact3.png}
\caption{Citra dari sumber garis diperoleh dari pergerakan detektor dari jarak (A) 30 cm ke (B) 5 cm terhadap permukaan kolimator. Pemosisian hasil citra yang bagus terlihat pada (C) dan artefak akibat pergeseran sekitar 2-mm terlihat pada (D)\cite{O'Connor1}}
\label{artifact3}
\end{figure}

Artefak pada SPECT umumnya terjadi pada rekonstruksi citra. Penggunaan kamera gamma dalam sistem SPECT akan mempengaruhi algoritma SPECT dalam melakukan rekonstruksi citra yang kemungkinan besar mengakibatkan munculnya artefak. Kesalahan perhitungan cacahan pada rekonstruksi dengan menggunakan \textit{filtered back projection} dan metode iteratif juga memunculkan artefak. Masalah sistematis tersebut harus dapat ditangani dan dioptimisasi dengan melakukan uji kelayakan sistem pencitraan SPECT.\cite{O'Connor1}\cite{O'Connor2}

\begin{figure}[h]
\includegraphics[scale=1]{Gambar/artifact4.png}
\caption{Citra transaksial phantom Jaszczak bagian \textit{hot rod} dengan rekonstruksi (A) koreksi COR, dan (B) COR error 3.2-mm yang mengakibatkan artefak}\cite{O'Connor1}
\label{artifact4}
\end{figure}

\begin{figure}[h]
\includegraphics[scale=0.75]{Gambar/artifact5.png}
\caption{Citra transaksial phantom Jaszczak bagian \textit{cold rod} diperoleh dengan menggunakan kolmator \textit{fan-beam} dan rekonstruksi (A) koreksi panjang focal, dan (B) Panjang focal diset tak hingga menghasilkan artefak}\cite{O'Connor1}
\label{artifact5}
\end{figure}

\section{Monte Carlo}
Kasus dalam kedokteran nuklir seperti peluruhan radioaktif dengan emisi energi, interaksi dengan material, pendeteksian foton, dsb merupakan fenomena keacakan (\textit{random}). Salah satu metode penyelesaian kasus random tersebut adalah dengan menggunakan metode Monte Carlo. Metode ini menjelaskan secara statistikal dengan menggunakan \textit{random numbers} sebagai variabel dasar dalam melakukan simulasi dari situasi spesifik. Dalam simulasi penggunaan random sampling yang menjelaskan proses fisika secara langsung dan akurat ditinjau dari model \textit{probability density functions} (pdfs).

Untuk error statistik, banyaknya jumlah simulasi dari kejadian interaksi (seperti foton dan jalur elektron) menjadi salah satu parameter penting dalam mengestimasi perhitungan. Walaupun dikatakan menggunakan \textit{random numbers}, sebenarnya variabel yang digunakan tidak sepenuhnya acak. Variabel tersebut lebih tepat disebut sebagai \textit{pseudo-random numbers}.

Kuantitas pdfs merupakan salah satu parameter penting dalam melihat informasi interaksi foton yang terjadi. Definisi dari pdfs adalah fungsi integral ternormalisasi dalam daerah [a,b]. Kumulasi pdfs dibentuk dari persamaan:

\begin{eqnarray}
cpdfs(x)=\int_{a}^{x}pdf(x' dx')  \label{eq9}
\end{eqnarray}

dengan $x$ sebagai sampel random terdistribusi uniform pada range tertentu. Perhitungan cpdf tidak selalu praktis dalam penggunaan Monte Carlo, sehingga ditambah metode rejeksi sebagai pendukung dalam melakukan simulasi. Langkah-langkah metode tersebut diantaranya:

\begin{enumerate}
\item Perhitungan $pdf_{A}(x)$ dalam range [a,b].
\item Perhitungan $pdf_{B}(x)$ untuk metode rejeksi yang dinormalisasi sehingga nilai maksimum $pdf_{B}(x)$ sama dengan nilai $pdf_{A}(x)$.
\item Pilih $x$ dari $pdf_{A}(x)$ secara random dengan menggunakan metode distribusi fungsi.
\item Masukkan metode rejeksi dari nilai $pdf_{B}(x)$ setelah melakukan metode distribusi fungsi.
\end{enumerate}

Sampling interaksi foton akibat hamburan dan absorpsi foton dapat ditinjau dari data \textit{cross-section}. Interaksi foton yang terjadi dapat berupa fotolistrik, hamburan Compton, dan \textit{pair-production}.

Panjang jalur foton dikalkulasi untuk menentukan apakah foton keluar dari VOI (\textit{Volume of Interest}) atau tidak. Jarak ini bergantung pada energi foton, densitas material dan komposisi material. Fungsi probabilitas diberikan:

\begin{eqnarray}
p(x)dx=\mu e^{-\mu x} dx \label{eq10}
\end{eqnarray}

Dimana $\mu$ adalah koefisien atenuasi material. Dari persamaan (\ref{eq10}) diperoleh kumulasi pdfs:

\begin{eqnarray}
cpdf(d)=\int_{0}^{d}\mu e^{-\mu} dx = [-e^{-\mu x}]_0^d=1-e^{-\mu d} \label{eq11}
\end{eqnarray}

$d$ merupakan jalur foton yang dilalui. Probabilitas jalur foton tersebut diperoleh:

\begin{eqnarray}
P(d)=1-e^{-\mu d} \label{eq12}
\end{eqnarray}

Jika diambil \textit{random number uniform} $R$ kemudian disubstitusi pada $P(d)$:

\begin{eqnarray}
R&=&1-e^{-\mu d} \nonumber \\
d&=&-\frac{1}{\mu}\ln(1-R)=-\frac{1}{\mu}\ln(R) \label{eq13}
\end{eqnarray}

Sampel dari panjang jalur foton dan arah pada koordinat kartesian dicek dengan perhitungan pada titik akhir apakah foton keluar dari VOI atau tidak. Dengan menggunakan koordinat baru $(x',y',z')$ pada koordinat kartesian dimana:

\begin{eqnarray}
x'&=&x+d\cdot u' \label{eq14} \\
y'&=&y+d\cdot v' \label{eq15} \\
z'&=&z+d\cdot w' \label{eq16}
\end{eqnarray}

Asumsikan $\theta$ dan $\phi$ adalah sudut polar dan azimuthal sistem koordinat kartesian, dan $\Theta$ dan $\Phi$ adalah sudut polar dan azimuthal yang mendefinisikan perubahan arah, maka:

\begin{eqnarray}
u'&=&\cos\theta\cdot u+\sin\theta[\cos\phi\cdot w\cdot u-\sin\theta\sin\phi\cdot v]/\sqrt{1-w^2} \label{eq17} \\
v'&=&\cos\theta\cdot v+[\sin\theta\cos\phi\cdot w\cdot v+\sin\theta\sin\phi\cdot v]/\sqrt{1-w^2} \label{eq18} \\
w'&=&\cos\theta\cdot w-\sin\theta\cos\phi\cdot\sqrt{1-w^2} \label{eq19}
\end{eqnarray}

Jalur foton akan mengalami interaksi-interaksi tertentu. Untuk mengetahui interaksi foton tersebut ditinjau dari koefisien atenuasi:

\begin{eqnarray}
\mu=\tau+\sigma_{inc}+\sigma_{coh}+\kappa \label{eq20}
\end{eqnarray}

Dengan $\tau$ menunjukkan interaksi foton, $\sigma_{inc}$ menunjukkan interaksi Compton, $\sigma_{coh}$ menunjukkan interaksi koheren, dan $\kappa$ menunjukkan \textit{pair-production}. Persamaan (\ref{eq20}) digunakan metode distribusi fungsi dan sampel \textit{random number uniform} $R$ serta beberapa kondisi yang menunjukkan interaksi foton yang terjadi.

\begin{enumerate}
\item $R<\tau/\mu$ merupakan interaksi fotolistrik.
\item $R<(\tau+\sigma_{inc})/\mu$ merupakan interaksi Compton.
\item $R<(\tau+\sigma_{inc}+\sigma_{coh})/\mu$ merupakan interaksi koheren.
\item Jika $R$ tidak memenuhi nilai 3 kondisi diatas, maka terjadi \textit{pair-production} dengan energi foton $>$ 1022 MeV.
\end{enumerate}

Energi foton dapat terabsorbsi sempurna oleh orbital elektron, sehingga jejak foton dihentikan dan energi tercatat. Namun deteksi energi foton dapat saja terjadi karena karakteristik sinar-X sekunder dan elektron Auger yang teremisi. Deposit energi yang datang pada material merupakan energi foton dikurangi energi ikat untuk melepaskan elektron.

Koreksi hamburan yang diperhitungkan terdiri dari hamburan foton inkoheren dan hamburan foton koheren. Untuk hamburan foton inkoheren mengikuti persamaan:

\begin{eqnarray}
h\nu'=\frac{h\nu}{1+(h\nu/m_oc^2)(1-\cos\theta)} \label{eq21}
\end{eqnarray}

Untuk meninjau sampel energi dan arah hamburan Compton menggunakan algoritma \textit{cross-section}:

\begin{eqnarray}
d\sigma_{KN}^e=\frac{r_e^2}{2}\left(\frac{h\nu'}{h\nu}\right)^2\left(\frac{h\nu}{h\nu'}+\frac{h\nu'}{h\nu}-\sin^2\theta\right)d\Omega \label{eq22}
\end{eqnarray}

Persamaan (\ref{eq22}) disebut juga persamaan Klein-Nishina. Parameter \textit{cross-section} pada persamaan (\ref{eq22}) dapat ditinjau dengan algoritma metode rejeksi:

\begin{eqnarray}
\frac{d\sigma_{incoh}}{d\Omega}=\frac{d\sigma_{KN}}{d\Omega}\cdot S(x,Z)=\frac{d\sigma_{KN}}{d\Omega}\cdot \frac{S(x,Z)}{S_{max}(x,Z)}\cdot K(h\nu,Z) \label{eq23}
\end{eqnarray}

$S(x,Z)$ merupakan fungsi hamburan inkoheren terhadap transfer momentum ($x$) dan nomor atom ($Z$). $K(h\nu,Z)$ adalah konstanta dari Z dan energi terfiksasi. Parameter transfer momentum dihitung dan $\theta$ dianggap jika \textit{random number} $R<[S(x,Z)/S_{max}(x,Z)]$.

Hamburan foton koheren ditinjau dari interaksi antar foton dan elektron dimana arah foton berubah tanpa kehilangan energi. Hamburan tersebut mengikuti perhitungan \textit{cross-section} Thomson:

\begin{eqnarray}
\frac{d\sigma}{d\Omega}=\frac{r_0^2}{2}(1+\cos^2\theta)[f^2(x,Z)]d\theta d\varphi \label{eq24}
\end{eqnarray}

Probabilitas foton terhambur pada interval $d\theta$ disekitar $\theta$:

\begin{eqnarray}
P(\theta)d\theta&=&K(h\nu,Z)\cdot G(\theta)\cdot f(x^2,Z) \label{eq25} \\
f(x^2,Z)&=&\frac{F^2(x,Z)}{\int_{0}^{x_{max}^2}F^2(x,Z)dx^2} \label{eq26}
\end{eqnarray}

$G(\theta)$ adalah fiksasi range foton dan $f(x^2,Z)$ adalah sampel prekalkulasi fungsi distribusi. Nilai sudut hamburan dapat dihitung jika relasi $R<G(\theta)$ terpenuhi.

Simulasi Monte Carlo dalam pengerjaannya membutuhkan banyak waktu, sehingga diperlukan pereduksian varians. Hal ini dilakukan untuk meningkatkan efisiensi simulasi dengan meninjau internal dosimetri dan proses biologis masuknya bahan radioaktif ke dalam suatu material. Teknik tersebut berdasarkan perhitungan foton dengan bobot.

\begin{figure}[h]
\includegraphics[scale=1]{Gambar/mc.png}
\caption{Tahap dasar simulasi interaksi foton pada volume yang ditentukan\cite{Ljungberg}}
\label{mc}
\end{figure}													%
%																%
%BAB 3 - METODOLOGI												%
% +++++++++++++++++++++++++++
%
% BAB 3: Metodologi
%
% +++++++++++++++++++++++++++

\chapter{METODE PENELITIAN}
\section{Alat dan Bahan}
Pada penelitian ini menggunakan beberapa alat dan bahan sebagai berikut.

\begin{itemize}
\item SPECT/CT Siemens \\
SPECT/CT dengan varian Siemens Symbia T series Intevo 16 di rumah sakit kanker Dharmais akan digunakan sebagai alat pencitraan. Sistem SPECT/CT tersebut terdiri dari 2 kepala kamera gamma yang sudah dihubungkan dengan komputer Syngo Siemens untuk melakukan akuisisi data dan rekonstruksi citra.

\item Technetium-99m (\textsuperscript{99m}Tc) \\
Radionuklida \textsuperscript{99m}Tc memiliki waktu paruh selama 6 jam dengan energi gamma yang dipancarkan sekitar 140.51 keV. Karakteristik dari \textsuperscript{99m}Tc tidak memancarkan partikel beta, konstanta sinar gamma spesifik sekitar 0.076 mrem/jam pada 1 meter per 1 mCi, dan aktivitas spesifiknya adalah 5,243,820 Ci/gr untuk \textsuperscript{99m}Tc murni dan $3.4 \times 10^6$ Ci/gr untuk \textsuperscript{99m}Tc bentuk pertechnetate.
\item Fantom Jazsczak \\
Model fantom Jazszcak yang akan digunakan adalah model \textit{Deluxe Flangeless} standar yang terdiri dari diameter silinder terstandarisasi dan beberapa sisipan. Sisipan pada fantom terdiri dari enam bola padat, sisipan \textit{Cold Rods} dan kapiler sebagai tempat dalam meletakkan sumber radionuklida dalam fantom. Ukuran diameter \textit{Cold Rods} adalah; 4.8 mm, 6.4 mm, 7.9 mm, 9.5 mm, 11.1 mm, 12.7 mm, dan ukuran diameter bola padat adalah; 9.5 mm, 12.7 mm, 15.9 mm, 19.1 mm, 25.4 mm, 31.8 mm.

\begin{figure}[h]
\includegraphics[scale=0.05]{Gambar/jaszsczak.jpg}
\caption{Fantom Jaszczak}
\label{jaszsczak}
\end{figure}

\item Handscoon \\
Dalam melakukan preparasi, yaitu ketika pengisian fantom, menggunakan handscoon (sarung tangan medis) agar lebih steril.

\item Syringe \\
Pengambilan dan pemasukkan air menggunakan syringe 10 mL dan 20 mL, sedangkan dalam memasukkan bahan radioaktif $^{99m}$Tc menggunakan syringe 10 mL.

\item Surveymeter \\
Surveymeter digunakan dalam mengukur paparan ketika melakukan pengisian fantom.

\item Air Mineral \\
Air digunakan sebagai medium dalam pencampuran bahan radioaktif $^{99m}$Tc pada fantom. Banyaknya air yang digunakan $\pm$ 36.5 mL.

\item Perangkat Lunak \\
Pengolahan citra untuk memperoleh data kuantitatif menggunakan software imageJ. Untuk melakukan simulasi Monte Carlo digunakan beberapa perangkat lunak seperti GATE (\textit{Geant4 Application for Tomographic Emission}) berbasis GEANT4. GATE dikembangkan dalam melakukan simulasi numerikal yang berkaitan dengan pencitraan medis dan radioterapi, terutama untuk simulasi emisi tomografi seperti PET, SPECT, CT.
\end{itemize}

\section{Metode Eksperimen}
Penelitian ini menggunakan dua metode, yaitu eksperimen dan simulasi. Pengukuran eksperimen sistem SPECT dilakukan dengan menggunakan kolimasi LEHR (\textit{Low Energy High Resolution}), variasi sudut $180^\circ$ tiap $2.8^\circ$, jumlah sampel linear per akuisisi 64, ukuran matriks citra $128 \times 128$. Langkah-langkah dalam melakukan eksperimen dilakukan berdasarkan rekomendasi ACR (\textit{American College of Radiology}).

\begin{itemize}
\item Pengisian Fantom Jaszczak \\
Akuisisi dilakukan dengan $^{99m}$Tc sebesar 10-20 mCi berdasarkan rekomendasi Siemens, dan penambahan aktivitas sebesar 20-40 mCi jika citra yang diperoleh tidak memenuhi kriteria. Fantom diisi terlebih dahulu dengan air hingga penuh sekitar 36.5 mL. Pengisian air tersebut bertujuan untuk meminimalisir gelembung-gelembung udara yang muncul pada fantom. Kemudian untuk pencampuran $^{99m}$Tc dilakukan dengan mengurangi air 1/5 bagian air pada fantom, setelah itu dimasukkan $^{99m}$Tc sesuai dengan aktivitas yang diperlukan. Setelah dicampur, tutup fantom ditutup dengan baik (agar tidak bocor) dan dikocok terlebih dahulu agar $^{99m}$Tc dan air tercampur merata. Kemudian dimasukkan kembali air hingga penuh dan gelembung-gelembung udara hilang, lalu dikocok ulang untuk memastikan $^{99m}$Tc dan air tercampur dengan baik. Setelah dikocok, fantom dimasukkan ke dalam kantong plastik dengan tujuan meminimalisir kebocoran yang terjadi.

\item Pemosisian Fantom Jaszzczak \\
Penempatan fantom Jaszczak ketika akuisisi sangat penting dalam memperoleh kualitas citra tinggi. Buruknya posisi fantom dengan sumbu rotasi akan mengakibatkan penurunan resolusi pada citra bagian \textit{cold rod}. Beberapa hal yang diperhatikan dalam pemosisian fantom adalah posisi bola padat terbesar berada pada 90$^\circ$ dari detektor, fantom berada pada pusat FOV (\textit{Field of View}), fantom diposisikan dalam keadaan seimbang dan tidak goyah (jika memungkinkan diberi pengganjal), dan rotasi radius detektor diatur sedekat mungkin dengan fantom.

\item Akuisisi Data \\
Template pengaturan pada SPECT secara default diatur untuk sumber radioaktif $^{99m}$Tc. Parameter SPECT untuk akuisisi adalah 64 frame untuk 180$^\circ$ dengan zoom citra 1.45. Jarak detektor 1 pada bed diatur 2.67 cm dan detektor 2 pada bed diatur 24.5 cm (berdasarkan prinsip penempatan detektor sedekat mungkin dengan fantom). posisi bed secara vertikal diatur sebesar 12 cm. Dilakukan proses scan dengan pengaturan \textit{time per view} 15 s, 20 s, 30 s, dan 40 s. Akuisisi dilakukan terus dengan memulai aktivitas tertinggi (20 mCi, 30 mCi, atau 40 mCi) hingga meluruh 10 mCi.

\item Processing Data \\
Citra SPECT yang diperoleh direkonstruksi dengan menggunakan \textit{Filtered Back Projection} dengan filter rekonstruksi \textit{Shepp-Logan-Hanning} dan frekuensi cut-off 1 $\times$ Nyquist. Slice citra yang tampil adalah 64 slice dengan zoom 1.45. Koreksi atenuasi diatur sebesar 0.11 cm$^{-1}$. Citra bagian uniformitas dipilih 6 slice, untuk citra bagian kontras 2 slice, dan citra bagian resolusi spasial 16 slice. Citra per bagian slice tersebut dikomposit (digabung) yang selanjutnya dianalisis dengan melihat line profil (untuk bagian uniformitas) dan melakukan ROI (untuk bagian kontras).

\item Perhitungan statistikal noise \\
Setelah melakukan rekonstruksi citra dengan sistem SPECT, hasil data piksel yang ada di komputer kemudian dilakukan perhitungan noise secara statistik. Estimasi noise diperoleh dengan menghitung densitas cacahan rata-rata sampel ($Y_{avg}$) dengan persamaan:

\begin{eqnarray}
Y_{avg} = \frac{1}{N} \sum_{i=1}^{N} y_i \label{eq9}
\end{eqnarray}

Dimana \textit{N} adalah jumlah piksel pada ROI (\textit{Region of Interest}) dekat titik tengah citra rekonstruksi bagian silinder uniform fantom SPECT, dan $y_i$ adalah jumlah cacahan  per piksel ke-\textit{i}. Dari hasil tersebut dihitung standar deviasi:

\begin{eqnarray}
S_{dev} = \frac{1}{\sqrt{N-1}}\sqrt{\sum_{i=1}^{N}(y_i - Y_{avg})^2} \label{eq10}
\end{eqnarray}

Dan rms sampel noise dalam persen diperoleh:

\begin{eqnarray}
\%rms = (S_dev/Y_{avg})\ast 100 \label{eq11}
\end{eqnarray}

\item Kontras citra \\
Rekonstruksi citra dari fantom yang menggunakan sisipan bola padat atau \textit{cold rods} akan diperoleh pengukuran kuantitatif kontras ($C_{image}$) untuk sisipan bola:

\begin{eqnarray}
C_{image} = \left|\frac{(Counts\cdot {Pixel}^{-1})_{sphere}-(Counts\cdot {Pixel}^{-1})_{background}}{(Counts\cdot {Pixel}^{-1})_{background}}\right| \label{eq12}
\end{eqnarray}

Untuk \textit{cold rods}:

\begin{eqnarray}
C_{image} = \left|\frac{(Counts\cdot {Pixel}^{-1})_{cold rods}-(Counts\cdot {Pixel}^{-1})_{background}}{(Counts\cdot {Pixel}^{-1})_{background}}\right| \label{eq13}
\end{eqnarray}

\item Pendeteksian artefak \\
Investigasi artefak dilakukan pengukuran inhomogenitas dengan menggunakan lempengan tembaga ukuran 10 $\times$ 10 $mm^2$ dengan ketebalan 0.3 $mm$, 0.5 $mm$, 0.7 $mm$, 0.9 $mm$, dan 1 $mm$. Tembaga tersebut ditempelkan pada kolimator sehingga mensimulasikan adanya kerusakan PMT pada detektor. Hasil scan tersebut memungkinkan munculnya ring artefak.
\end{itemize}

\section{Simulasi Monte Carlo}
Untuk metode simulasi digunakan perangkat lunak GATE dengan langkah awal mendesain sistem scan SPECT dan fantom Jazszcak, lalu membuat protokol optimisasi, melakukan tes algoritma, memperoleh data kuantifikasi, melakukan koreksi hamburan, menganalisis data, dan terakhir rekonstruksi citra.
Salah satu code simulasi Monte Carlo yang digunakan adalah GATE. Code GATE merupakan salah satu dari program GEANT4 berbasis bahasa pemrograman C++ pada OS linux. Fitur dari simulasi GATE ini terdiri dari deskripsi akurat fenomena yang bergantung waktu seperti gerakan detektor, pernafasan pasien, gerakan kadiak, kinetika radiotracer, dsb. Pada aplikasi GATE terdapat perhitungan sistem matriks yang menggunakan rekonstruksi dan produksi data untuk evaluasi koreksi algoritma. Penggunaan GATE sebagai contoh dalam menentukan dosimetri, simulasi studi klinis dan fokus optimisasi algortima.

Fitur-fitur yang disediakan GATE untuk melakukan simulasi adalah:

\begin{enumerate}
\item Konstruksi geometri dari modalitas (SPECT atau PET) dalam melakukan simulasi, dan bentuk dari phanto yang akan digunakan.
\item Proses fisik yang terjadi dalam simulasi meliputi interaksi-interaksi yang terjadi pada foton atau elektron.
\item Sumber yang digunakan beserta data spesifikasinya.
\item Pergerakan dari foton dan elektron berupa jenis, kecepatan, atau komponen lainnya dalam melihat jalur interaksi sehingga dapat ditinjau \textit{cross-section} dan probabilitasnya. Durasi waktu akuisisi pengambilan data
\end{enumerate}

Output data dari GATE dalam bentuk ROOT, file ASCII, file binary, dan format data spesifik scanner yang disimulasikan.

Data parameter spesifikasi gamma kamera yang digunakan dan parameter dalam proses akuisisi SPECT akan digunakan dalam simulasi GATE. Parameter fantom Jaszczak dimasukkan pula ke dalam GATE sebagai desain fantom. Simulasi akan dilakukan dengan membuat desain eksperimen dalam keadaan  sesuai dengan eksperimen manual. Beberapa hal yang perlu diperhatikan dalam modelling bergantung waktu diantaranya:

\begin{enumerate}
\item Perubahan distribusi tracer terhadap waktu.
\item Gerakan detektor dalam akuisisi.
\item Peluruhan radioaktif.
\item \textit{Dead time} pada detektor.
\end{enumerate}

Prinsip dasar dari modelling waktu di GATE adalah menyesuaikan pemodelan peluruhan radioaktif berdasarkan skema peluruhan yang akan terjadi. Kemudian mendefinisikan variabel waktu ketika simulasi eksperimen sehingga data akan berubah berdasarkan perubahan waktu.

\begin{figure}[h]
\includegraphics[scale=0.5]{Gambar/simulation1.png}
\caption{Contoh model simulasi desain SPECT pada GATE}
\label{simulation1}
\end{figure}

\begin{figure}[h]
\includegraphics[scale=0.6]{Gambar/simulation2.png}
\caption{Modelling konstruksi geometri pada GATE dengan command script}
\label{simulation2}
\end{figure}

\begin{figure}[h]
\includegraphics[scale=0.7]{Gambar/simulation3.png}
\caption{Modelling peluruhan sumber radioaktif}
\label{simulation3}
\end{figure}

\begin{figure}[h]
\includegraphics[scale=0.7]{Gambar/simulation4.png}
\caption{Data output simulasi GATE}
\label{simulation4}
\end{figure}

Kemudian modelling respon dari sistem elektronik yang digunakan dalam simulasi pencitraan. Komponen-komponen dan cara kerja dari beberapa rangkaian digital perlu diketahui dalam melakukan processing data simulasi. Pada bagian ini diusahakan mereproduksi kurva \textit{count rate} secara akurat.

Untuk memudahkan desain detektor simulasi dapat dipecah menjadi dua bagian, yaitu transportasi foton pada objek dan transportasi foton pada detektor. Output dari transportasi foton pada objek direkam pada ruang fase. Teknik ini berguna dalam mengetes detektor yang didesain tanpa melakukan pengulangan.

Algoritma yang digunakan dalam simulasi mengikuti gambar \ref{mc}. Kemudian dalam pendeteksian artefak, jenis ring artefak yang akan ditinjau. Pada simulasi tembaga (Cu) atau timbal (Pb) ditempel pada kolimator sehingga dari hal tersebut akan melihat rekonstruksi dari ring artefak sehingga analisis dari artefak yang dihasilkan akan dibandingkan dengan pengukuran manual yang tidak menggunakan Cu dan Pb. Optimalisasi protokol dan algoritma yang digunakan mengikuti gambar (\ref{mc}).

\begin{figure}[h]
\includegraphics[scale=0.5]{Gambar/diagpen.png}
\caption{Diagram Alir Penelitian}
\label{diagpen}
\end{figure}													%
%																%
%BAB 4 - HASIL 													%
% +++++++++++++++++++++++++++
%
% BAB 4: Hasil
%
% +++++++++++++++++++++++++++

\chapter{HASIL DAN ANALISIS}

\section{Energi Statik dan Energi Rotasi}
 Dari hasil perhitungan (\ref{eq:euler-lagrange}) yang menghasilkan grafik fungsi profil pada gambar (\ref{fig:profileL26}) secara numerik, kita dapatkan nilai eksak dari perhitungan energi statik dengan meninjau persamaan (\ref{eq:L26static2}). Bilangan baryon yang dipakai sesuai dengan solusi BPS, dimana $E=|B|$, adalah $B=|n|=1$. Nilai $n$ tersebut merupakan solusi energi pada koordinat bola, yang disebut sebagai solusi \textit{hedgehog} pada solusi fungsi profil, dan dipakai untuk mendapatkan nilai minimum energi pada energi statik yang dihitung. Hasil energi dari fungsi profil adalah $E=1,086$, dimana nilai tersebut melebihi ikatan BPS sekitar $8\%$. Walaupun nilai tersebut tidak eksak sebagai syarat dari solusi BPS, dimana $E=1$, hasil tersebut dapat digunakan sebagai pendekatan hampir-BPS Skyrme karena memiliki tingkat \textit{error} yang lebih rendah daripada nilai eksak energi pada persamaan (\ref{eq:staticskyrme}), dimana $E=1,232$.

Jika kita tinjau persamaan (\ref{eq:hamiltonianL26}), hasil tersebut memiliki kesamaan pada kuantisasi Skyrmion dengan semua suku Lagrangian pada (\ref{eq:rotationhamiltonian}). Hanya definisi untuk momen inersia Skyrmion pada $U$, $V$, dan $W$ berbeda dimana suku Skyrme akan hilang karena kita telah menset nilai parameter $\beta$ dianggap sangat kecil. Dari hasil tersebut mengindikasikan bahwa persamaan Skyrmion (\ref{eq:L26}) menunjukkan karakteristik dari gerakan rotasi pada \textit{rigid-body} inti nuklir\cite{Bonenfant},\cite{Beaudoin},\cite{Marleau}. Namun dari kuantisasi tersebut, kita belum mengetahui karakteristik dari masing-masing proton dan neutron dari model inti, sehingga efek rotasi dan isorotasi dideskripsikan sebagai nukleon.

Hasil energi statik (\ref{eq:L26static1}) dan energi rotasi (\ref{eq:hamiltonianL26}), kita plot secara grafis kedua energi masing-masing dan energi total dari penjumlahan kedua energi dengan hubungan bilangan baryon $n$ dengan menggunakan solusi numerik dari fungsi profil $g(r)$ pada gambar \ref{fig:profileL26}.

Dari gambar (\ref{fig:energy}), energi statik berbanding lurus dengan $n$. Semakin besar $n$, maka semakin besar pula energi statik, sehingga energi ikat dari inti nuklir akan semakin besar pula dalam kasus model Skyrme. Sedangkan efek dari energi rotasi memiliki karakteristik yang berbanding terbalik dengan energi statik. Energi total dari energi statik dan energi rotasi memperlihatkan bahwa efek dari energi statik lebih besar daripada efek dari energi rotasi. Berdasarkan teori model nuklir, energi rotasi mendeskripsikan gerakan \textit{rigid body} pada inti nuklir dalam kasus mekanika klasik, dimana energi statik yang dihasilkan pada Lagrangian Skyrme memiliki peran dalam mempertahankan bentuk dari inti nuklir tersebut dengan \textit{momenta} energi seminimal mungkin.

\section{Nilai Konstanta Kopling}
Parameter konstanta kopling yang dicari adalah parameter konstanta $\lambda$ yang ada pada suku derivatif orde keenam $L_6$. Konstanta $\lambda $ tersebut merupakan konstanta tidak berdimensi yang muncul ketika kita menstabilkan persamaan Lagrangian Skyrmion yang diekspansikan. Dalam arti fisis, $\lambda$ mendeskripsikan inetraksi yang terjadi pada nukleon dalam Lagrangian.

Nilai konstanta kopling $\lambda$ dicari dari persamaan (\ref{eq:mass}), dimana $m$ yang dipakai ada dua, yaitu massa nukleon yang bernilai $938,9$ MeV dan massa delta $1232$ MeV. Hasil perhitungan dengan menggunakan kedua nilai massa tersebut diperlihatkan pada tabel (\ref{table:lambda}).
\begin{table}
\begin{center}
\begin{tabular}{||c| c| c||} 
\hline
Parameter & $\lambda_1$ & $\lambda_2$ \\ [0.5ex] 
\hline\hline
Massa Nukleon ($938,9$) MeV & $7,575\times 10^{-5}$ & $8,303\times 10^{-5}$ \\ 
\hline
Massa Delta ($1232$) MeV & $1,979\times 10^{-4}$ & $1,387\times 10^{-3}$ \\ [1ex]
\hline
\end{tabular}
\caption{Tabel hasil konstanta kopling ($\lambda$) dengan set massa nukleon dan massa delta.}
\label{table:lambda}
\end{center}
\end{table}
Kemudian dari tabel (\ref{table:lambda}), kita gunakan nilai $\lambda$ ke persamaan (\ref{eq:mass}). Hasil perhitungan massa nukleon dan massa delta untuk masing-masing $\lambda$ terlihat pada tabel (\ref{table:mass}).

\begin{table}
\begin{center}
\begin{tabular}{||c| c| c||} 
\hline
Konstanta Kopling ($\lambda$) & Massa Nukleon (MeV) & Massa Delta (MeV) \\ [0.5ex] 
\hline\hline
$7,575\times 10^{-5}$ & 938,9 (eksperimen) & 3582,13 \\
\hline
$8,303\times 10^{-5}$ & 938,9 (eksperimen) & 1011,74 \\
\hline
$1,979\times 10^{-4}$ & 605,981 & 1232 (eksperimen) \\
\hline
$1,387\times 10^{-3}$ & 1198,25 & 1232 (eksperimen) \\ [1ex]
\hline
\end{tabular}
\caption{Tabel hasil massa nukleon dan massa delta dengan memasukkan nilai parameter konstanta kopling yang diperoleh dari tabel (\ref{table:lambda}).}
\label{table:mass}
\end{center}
\end{table}

Jika kita lihat tabel (\ref{table:lambda}) dan tabel (\ref{table:mass}), nilai $\lambda$ yang mendekati dengan nilai massa delta untuk massa nukleon yang diset nilainya secara eksperimen adalah $\lambda_2$, karena massa delta yang diperoleh memiliki \textit{error} (kesalahan literatur) sekitar $17,9\%$. Untuk massa delta yang diset nilainya, nilai massa nukleon terbaik ketika kita menggunakan $\lambda_2$, dimana \textit{error} yang diperoleh adalah $27,8\%$. Jika kita buat rentang nilai terbaik $\lambda$ adalah $8,303\times 10^{-5}\leq\lambda\leq 1,387\times 10^{-3}$. Dengan mengambil nilai $\lambda$ terbaik dari rentang tersebut dengan menggunakan metode \textit{mean}, maka kita akan memperoleh $\lambda_{terbaik}=1,108\times 10^{-3}$, sehingga $m_{nukleon}=1075,6$ dengan \textit{error} $14,6\%$ dan $m_{delta}=1122,8$ dengan \textit{error} $8,8\%$.

Cara lain untuk memperoleh nilai $\lambda$ adalah dengan menggunakan persamaan dari penjumlahan massa nukleon dan massa delta.
\begin{eqnarray}
m_{nukleon}+m_{delta}&=&(M_{statik}+M_{rotasi})_{nukleon}+(M_{statik}+M_{rotasi})_{delta} \nonumber \\
&=&\left(\frac{3f_\pi^3\lambda}{\sqrt{2}}\right)^{\frac{1}{2}}\pi\mathcal{E}_{statik}+\left(\frac{1}{2}\left[\frac{j(j+1)}{V_{11}}+\frac{i(i+1)}{U_{11}}\right.\right. \nonumber \\
&&\left.\left.+\left(\frac{1}{U_{33}}-\frac{1}{U_{11}}-\frac{n^2}{V_{11}}\right)k_3^2\right]\right)_{nukleon} \nonumber \\
&&+\left(\frac{1}{2}\left[\frac{j(j+1)}{V_{11}}+\frac{i(i+1)}{U_{11}}\right.\right. \nonumber \\
&&\left.\left.+\left(\frac{1}{U_{33}}-\frac{1}{U_{11}}-\frac{n^2}{V_{11}}\right)k_3^2\right]\right)_{delta} \label{eq:summass}
\end{eqnarray}
Dari persamaan (\ref{eq:summass}) disederhanakan menjadi persamaan:
\begin{eqnarray}
63904,2\lambda^{\frac{1}{2}}+0,0026\lambda^{-\frac{3}{2}}-2170,9=0 \label{eq:simplemass}
\end{eqnarray}
Solusi yang diperoleh dari persamaan (\ref{eq:simplemass}) adalah $\lambda_3=1,531\times 10^{-4}$ dan $\lambda_4=1,077\times 10^{-3}$, dengan $\lambda_4$ sebagai $\lambda_{terbaik}$ dan $m_{nukleon}+m_{delta}=2170,9$.													%
%																%
%BAB 5 - KESIMPULAN 											%
% +++++++++++++++++++++++++++
%
% BAB 5: Kesimpulan dan saran
%
% +++++++++++++++++++++++++++

\chapter{KESIMPULAN}

\section{Kesimpulan}
Hasil perhitungan energi statik pada menggunakan model Skyrme $SU(2)$ dengan Lagrangian suku $L_2$ dan $L_6$ memiliki solusi yang mendekati BPS, dimana solusi tersebut memiliki error yang melebihi sekitar $8\%$. Untuk fungsi profil yang menghasilkan solusi mendekati BPS tersebut, kita dapat menggunakannya dalam menghitung energi rotasi selanjutnya. 

Energi rotasi yang diperoleh memiliki bentuk yang sama dengan perhitungan sebelumnya yang menggunakan suku semua suku, dengan definisi momen inersia yang berbeda karena kehilangan suku $L_4$. Energi rotasi juga mendeskripsikan gerakan rotasi dan isorotasi pada \textit{rigid body} dari inti nuklir pada model Skyrme, walaupun nukleon yang dideskripsikan pada model ini belum kami pisahkan karakteristik dari proton dan neutron. 

Energi statik dan energi rotasi yang diperoleh dengan menggunakan solusi BPS dapat kita pelajari deskripsi energi ikat, dimana pada perhitungan sebelumnya yang menggunakan suku $L_6$ dan $L_0$ mendeskripsikan arti fisis pada inti nuklir untuk mempelajari energi ikat. Pengaruh dari energi statik lebih besar daripada energi rotasi pada model Skyrme, sehingga pernyataan tersebut sesuai dengan kondisi sistem \textit{rigid body} yang ada pada teori model nuklir.

Untuk nilai konstanta kopling $\lambda$ yang diperoleh, kita peroleh nilai terbaik dengan $\lambda=1,077\times 10^{-3}$ karena memiliki tingkat \textit{error} terendah pada perhitungan massa nukleon dan massa delta. Nilai konstanta kopling tersebut nantinya dibandingkan dengan nilai konstanta kopling pada perhitungan \textit{fitting} energi ikat model Skyrme.

\section{Saran}

Dari hasil yang disajikan dalam skripsi ini, studi energi ikat untuk model Skyrme masih sangatlah kurang, mengingat bahwa energi ikat pada inti nuklir tidak hanya terdiri dari energi statik dan energi rotasi, dan \textit{fitting} data untuk memperoleh parameter model tersebut belum dilakukan. Pengaruh dari energi coulomb dan energi isospin \textit{symmetry breaking} yang telah dihitung pada Lagrangian Skyrmion belum dilakukan. Bahkan energi tegangan permukaan pada model Skyrme belum pernah dibahas sebelumnya, sehingga penelitian lebih lanjut perlu dilakukan untuk mempelajari energi ikat yang lebih lanjut pada model Skyrme $SU(2)$.													%
%																%
%REFERENSI														%
% +++++++++++++++++++++++++++
%
% REFERNSI
%
% +++++++++++++++++++++++++++

% Ubah Tampilan Nama
\renewcommand{\bibname}{DAFTAR REFERENSI}
% Memasukkan ke daftar isi
\addcontentsline{toc}{chapter}{\numberline{}Daftar Referensi}

%Daftar referensi
\begin{thebibliography}{99}

%\cite{Hirtl}
\bibitem{Hirtl} 
  A. Hirtl, H. Bergmann, T. Beyer, M. Figl and J. Hummel,
  ''Technical Note: Fully-automated anaylisis of Jaszczak phantom measurements as part of routine SPECT quality control,''
  Med Phys. 2017 May;44(5):1638-1645. doi: 10.1002/mp.12150. Epub 2017 Apr 12.
  
%\cite{Ferreira}
\bibitem{Ferreira} 
  F. C. L. Ferreira and D. N. Souza,
  ''Liver phantom for quality control and training in nuclear medicine,''
  Nuclear Instruments and Methods in Physics Research A, Volume 652, Issue 1, 1 October 2011, Pages 791-793.

%\cite{Alehyani}
\bibitem{Alehyani}
  S. H. A. ALehyani
  ''Application of single photon emission computed tomography (SPECT) parameters for bone scintigraphy''
  Journal of King Saud University (Science), Volume 21, Issue 2, July 2009, Pages 109-117.

%\cite{Roshan}
\bibitem{Roshan} 
  H. R. Roshan, B. Mahmoudian, E. Gharepapagh, A. Azarm and J. P. Islamian,
  ''Collimator and energy window optimization for \textsuperscript{90}Y bremsstrahlung SPECT imaging: A SIMIND Monte Carlo study,''
  Applied Radiation and Isotopes, Volume 108, February 2016, Pages 124–128.

%\cite{D'Arienzo1}
\bibitem{D'Arienzo1} 
  M. D'Arienzo, M. L. Cozzella, A. Fazio, P. De Felice, G. Iazzarino, M. D'Andrea, S. Ungania, M. Cazzato, K. Schmidt, S. Kimiaei and L. Strigari,
  ''Quantitative \textsuperscript{77}Lu SPECT imaging using advanced correction algorithms in non-reference geometry,''
  Physica Medica, Volume 32, Issue 12, December 2016, Pages 1745-1752.

%\cite{Zhou}
\bibitem{Zhou} 
  X. Zhou, M. Yun, X. Cao, S. Liu, L. Wang, X. Huang and L. Wei,
  ''Evaluation of imaging protocol for ECT based on CS image reconstruction algorithm,''
  Chinese Physics C, Volume 38, Number 4, (2014).

%\cite{Wolf}
\bibitem{Wolf} 
  P. A. Wolf, J. H. Jorgensen, T. G. Schmidt and E. Y. Sidky,
  ''Few-view single photon emission computed tomography (SPECT) reconstruction based on a blurred piecewise constant object model,''
  Physics in Medicine \& Biology, Volume 58, Number 16, (2013).

%\cite{Loudos1}
\bibitem{Loudos1} 
  G. K. Loudos,
  ''An efficient analytical calculation of probability matrix in 2D SPECT,''
  Computerized Medical Imaging and Graphics, Volume 32, Issue 2, March 2008, Pages 83-94.

%\cite{Bushberg}
\bibitem{Bushberg} 
  J. T. Bushberg et al.,
  ''The Essensial Physics of Medical Imaging Third Edition,''
  (Lippincot Williams \& Wilkins, 2012).

%\cite{O'Connor1}
\bibitem{O'Connor1} 
  M. K. O'Connor,
  ''Instrument- and computer-related problems and artifacts in nuclear medicine,''
  Seminars in Nuclear Medicine, Volume 26, Issue 4, October 1996, Pages 256-277.

%\cite{O'Connor2}
\bibitem{O'Connor2} 
  M. K. O'Connor and B. J. Kemp,
  ''Single-Photon Emission Computed Tomography/Computed Tomography: Basic Instrumentation and Innovations,''
  Seminars in Nuclear Medicine, Volume 36, Issue 4, October 2006, Pages 258-266.
  [https://doi.org/10.1053/j.semnuclmed.2006.05.005].
  
%\cite{D'Arienzo2}
\bibitem{D'Arienzo2} 
  M. D'Arienzo, M. Cazzato, M. L. Cozzella, M. Cox, M. D'Andrea, A. Fazio, A. Fenwick, G. Iazzarino, L. Johansson, L. Strigari, S. Ungania and P. De Felice,
  ''Gamma camera calibration and validation for quantitative SPECT imaging with \textsuperscript{177}Lu,''
  Applied Radiation and Isotopes, Volume 112, June 2016, Pages 156-164.

%\cite{Ng}
\bibitem{Ng}
  A. H. Ng, K. H. Ng, H. Dharmendra and A. C. Perkins,
  ''A low-cost phantom for simple routine testing of single photon emission computed tomography (SPECT) cameras'' 
  Applied Radiation and Isotopes, Volume 67, Issue 10, October 2009, Pages 1864-1868.
  [https://doi.org/10.1016/j.apradiso.2008.10.010].

%\cite{Greffier}
\bibitem{Greffier} 
  J. Greffier, F. Macri, A. Labri, A. Fernandez, F. Pereira, C. Mekkaoui and J. -P. Beregi,
  ''Dose reduction with iterative reconstruction in multi-detector CT: What is the impact on deformation of circular tructures in phantom study?''
  Diagnostic and Interventional Imaging, Volume 97, Issue 2, February 2016, Pages 187-196.

%\cite{Qi}

\bibitem{Qi} 
  J. Qi and R. H. Huesman,
  ''Effect of errors in the system matrix on maximum a posteriori image reconstruction''
  Physics in Medicine \& Biology, Volume 50, Number 14.
  [http://iopscience.iop.org/article/10.1088/0031-9155/50/14/007/meta]

%\cite{Buvat}
\bibitem{Buvat} 
  I. Buvat,
  ''Using simulations of the detector performance for enhanced image reconstruction in molecular imaging''
  Nuclear Instruments and Methods in Physics Research A, Volume 809, 11 February 2016, Pages 89-95.

%\cite{Ljungberg}
\bibitem{Ljungberg} 
  M. Ljungberg et al.,
  ''Monte Carlo Calculations in Nuclear Medicine: Applications in Diagnostic Imaging, Second Edition,''
  (Taylor \& Francis, 1998).

%\cite{Loudos2}
\bibitem{Loudos2} 
  G. K. Loudos,
  ''Monte Carlo simulations in Nuclear Medicine''
  AIP Conference Proceedings, Volume 958, November 2017, Pages 147-150.

%\cite{Strulab}
\bibitem{Strulab} 
  D. Strulab et al,
  ''GATE (Geant4 Application for Tomographic Emission): a PET/SPECT general-purpose simulation platform''
  Nuclear Physics B - Proceedings Supplements, Volume 125, 2013, Pages 75-59.

%\cite{Assie}
\bibitem{Assie} 
  K. Assie et al,
  ''Monte Carlo simulation in PET and SPECT instrumentation using GATE''
  Nuclear Instruments and Methods in Physics Research Section A: Accelerators, Spectrometers, Detectors and Associated Equipment, Volume 527, 2004, Pages 180-189.

%\cite{Khosravi}
\bibitem{Khosravi} 
  H. R. Khosravi et al,
  ''Planar and SPECT Monte Carlo acceleration using a variance reduction technique in I$^{131}$ imaging''
  J. Radiat. Res, Volume 4, March 2007, Pages 175-182.

%\cite{Gilliam}
  \bibitem{Gilliam} 
  J. E. Gilliam et al,
  ''Monte-Carlo simulations and image reconstruction for novel imaging scenarios in emission tomography''
  Nuclear Instruments and Methods in Physics Research Section A: Accelerators, Spectrometers, Detectors and Associated Equipment, Volume 809, 2016, Pages 76-88.

\end{thebibliography}												%
% 																%
% ==============================================================%
%					 	   LAMPIRAN								%
% ==============================================================%
%																%
%Lampiran														%
% +++++++++++++++++++++++++++
%
% LAMPIRAN-LAMPIRAN
%
% +++++++++++++++++++++++++++

\appendix

\chapter{Topologi}
Topologi merupakan salah satu dari cabang matematika yang mempelajari karakteristik dari suatu objek dengan penggambaran fungsi kontinu. Materi topologi berhubungan dengan objek yang terdeformasi tanpa merusak kualitas dari objek tersebut, dapat dikatakan bentuk dari objek tersebut invarian terhadap suatu deformasi. Tinjauan kasus topologi pada objek tidak meliputi ukuran, bentuk, jarak, ataupun sudut, topologi hanya meninjau perubahan pada transformasi yang kontinu. Objek yang diregangkan, diremas, dan diputar pada bagian tertentu tidak merubah topologi, tetapi objek yang dirobek, digunting, dan ditempel pada objek lain merubah topologi. Topologi berkaitan dengan teori \textit{set} dalam matematika\cite{Handhika},\cite{Katok}.

Misalkan \textit{set} $X$ merupakan \textit{set} yang tidak kosong. Kumpulan \textit{subset} $T$ pada $X$ dikatakan topologi di $X$ jika:
\begin{enumerate}
\item $X$ dan \textit{set} kosong, $\emptyset$, bagian dari \textit{T}.
\item Gabungan semua angka berhingga atau tak hingga dalam \textit{set} di $T$ merupakan bagian juga dari $T$.
\item Irisan dari kedua \textit{set} di $T$ bagian juga dari $T$.
\end{enumerate}
Pasangan $(X,T)$ disebut sebagai ruang topologikal. \\
Contoh 1: \\
Misalkan $X=\{a,b,c,d,e,f\}$ dan $T_1=\{X,\emptyset,\{a\},\{c,d\}\{a,c,d\},\{b,c,d,e,f\}\}$, maka $T_1$ dapat dikatakan topologi terhadap $X$. \\
Contoh 2: \\
Misalkan $X=\{a,b,c,d,e\}$ dan $T_2=\{X,\emptyset,\{a\},\{c,d\}\{a,c,e\},\{b,c,d\}\}$, maka $T_2$ dapat dikatakan bukan topologi terhadap $X$ karena:
\begin{eqnarray}
\{c,d\}\cup\{a,c,e\}=\{a,c,d,e\}
\end{eqnarray}
Dapat dilihat bahwa gabungan antara dua anggota $T_2$ tidak termasuk kedalam $T_2$. \\
Contoh 3: \\
Misalkan $X=\{a,b,c,d,e,f\}$ dan $T_3=\{X,\emptyset,\{a\},\{f\}\{a,f\},\{a,c,f\}\{b,c,d,e,f\}\}$, maka $T_3$ dapat dikatakan bukan topologi terhadap $X$ karena:
\begin{eqnarray}
\{a,c,f\}\cap\{b,c,d,e,f\}=\{c,f\}
\end{eqnarray}
Dapat dilihat bahwa irisan antara dua anggota $T_3$ tidak termasuk kedalam $T_3$.

Apabila $X$ merupakan \textit{set} yang tidak kosong dan $T$ merupakan kumpulan masing-masing \textit{subset} dari $X$, maka $T$ dikatakan topologi diskrit pada $X$. Dan apabila $T=\{X,\emptyset\}$, maka $T$ disebut sebagai topologi indiskrit.

Istilah dari topologi yang mungkin akan sering ditemui adalah \textit{open sets}, \textit{closed sets}, dan \textit{clopen sets}. Misalkan $(X,T)$ merupakan ruang topologikal, maka setiap anggota dari $T$ dapat dikatakan \textit{open sets}. \textit{Subset} $S$ dikatakan sebagai \textit{closed set} pada $(X,T)$ jika \textit{subset} tersebut hanya sebagai pelengkap pada $X$, untuk \textit{subset} $X\setminus S$ dikatakan \textit{open} pada $(X,T)$. Pada contoh 1, bagian dari \textit{closed sets} adalah: $\emptyset,X,\{b,c,d,e,f\},\{a,b,e,f\},\{b,e,f\}$ dan $\{a\}$.

Sebuah \textit{set} dapat diakatan pula \textit{open} dan \textit{closed}, atau tidak keduanya, tergantung dari kasus yang ditinjau. Jika kita tinjau contoh 1:
\begin{enumerate}
\item \textit{Set} $\{a\}$ keduanya \textit{open} dan \textit{closed}.
\item \textit{Set} $\{b,c\}$ keduanya tidak \textit{open} dan \textit{closed}.
\item \textit{Set} $\{c,d\}$ adalah \textit{open}, tetapi tidak \textit{closed}
\item \textit{Set} $\{a,b,e,f\}$ adalah \textit{closed}, tetapi tidak \textit{open}.
\end{enumerate}
\textit{Set} yang keduanya adalah \textit{open} dan \textit{closed} disebut sebagai \textit{clopen sets}. Pada setiap ruang topologikal $(X,T)$, $X$ dan $\emptyset$ keduanya adalah \textit{clopen}. Pada topologi diskrit, semua \textit{subset} dari $X$ adalah \textit{clopen}.
Pada topologi indiskrit, \textit{clopen} hanya \textit{subset} dari $X$ dan $\emptyset$ \cite{Morris}.

\chapter{Grup $SU(2)$}
Tinjau grup $U(2)$:
\begin{eqnarray}
A&=&\begin{pmatrix} a & b \\ c & d \end{pmatrix} \nonumber \\
A^\dagger&=&\begin{pmatrix} a^* & b^* \\ c^* & d^* \end{pmatrix} ,
\end{eqnarray}
Karena matriks $A$ memenuhi sifat unitari, maka:
\begin{eqnarray}
A^\dagger A&=&\textbf{1} \\
a&=&a_R+ia_I \\
b&=&b_R+ib_I \\
c&=&c_R+ic_I \\
d&=&d_R+id_I
\end{eqnarray}
Sehingga dapat ditulis:
\begin{eqnarray}
A^\dagger A&=&\begin{pmatrix} a^* & b^* \\ c^* & d^* \end{pmatrix} \begin{pmatrix} a & b \\ c & d \end{pmatrix} \nonumber \\
&=&\begin{pmatrix} |a|^2+|c|^2 & a^*b+c^*d \\ b^*a+d^*c & |b|^2+|d|^2 \end{pmatrix}=\begin{pmatrix} 1 & 0 \\ 0 & 1 \end{pmatrix} ,
\end{eqnarray}
Dari kondisi diatas kita ketahui:
\begin{eqnarray}
|a|^2+|c|^2&=&|b|^2+|d|^2=1 \\
a^*b+c^*d&=&b^*a+d^*c ,
\end{eqnarray}
Dari matriks $A$ dapat diketahui:
\begin{eqnarray}
A^\dagger&=&A^{-1} \\
\det{A}&=&ad-bc \\
A^{-1}&=&\frac{1}{\det{A}}\begin{pmatrix} d & -b \\ -c & a \end{pmatrix}
\end{eqnarray}
Grup $SU(2)$ memiliki nilai determinan $|1|$, sehingga:
\begin{eqnarray}
\det{A}&=&1 \\
A^{-1}&=&\begin{pmatrix} d & -b \\ -c & a \end{pmatrix}=\begin{pmatrix} a^* & c^* \\ b^* & d^* \end{pmatrix} \\
A&=&\begin{pmatrix} a & b \\ -b^* & a^* \end{pmatrix} ; A^\dagger=\begin{pmatrix} a^* & -b \\ b^* & a \end{pmatrix} \\
A^\dagger A&=&\begin{pmatrix} |a|^2+|b|^2 & 0 \\ 0 & |a|^2+|b|^2 \end{pmatrix} ,
\end{eqnarray}
Tinjau kasus \textit{infinitesimal} dimana:
\begin{eqnarray}
A&=&\textbf{1}+id\alpha_jx_j+... \nonumber \\
\begin{pmatrix} a & b \\ -b^* & a^* \end{pmatrix}&=&\begin{pmatrix} 1 & 0 \\ 0 & 1 \end{pmatrix}+... \nonumber \\
A&=&\begin{pmatrix} (1+\frac{i}{2}a_{11}) & \frac{1}{2}(b_{12}+ia_{12}) \\ -\frac{1}{2}(b_{12}-ia_{12}) & (1-\frac{i}{2}a_{11}) \end{pmatrix} \nonumber \\
&=&\begin{pmatrix} 1 & 0 \\ 0 & 1 \end{pmatrix}+\frac{i}{2}a_{11}\begin{pmatrix} 1 & 0 \\ 0 & -1 \end{pmatrix}+\frac{i}{2}a_{12}\begin{pmatrix} 0 & 1 \\ 1 & 0 \end{pmatrix} \nonumber \\
&&+\frac{i}{2}b_{12}\begin{pmatrix} 0 & -i \\ i & 0 \end{pmatrix} \nonumber \\
&=&\textbf{1}+\frac{i}{2}a_{11}\sigma_z+\frac{i}{2}a_{12}\sigma_x+\frac{i}{2}b_{12}\sigma_y ,
\end{eqnarray}
Definisi dari $\sigma_i$ dengan $i=x,y,z$ adalah matriks Pauli, yang merupakan operator spin pada tinjauan mekanika kuantum.
\begin{eqnarray}
\sigma_x=\begin{pmatrix} 0 & 1 \\ 1 & 0 \end{pmatrix} , \sigma_y=\begin{pmatrix} 0 & -i \\ i & 0 \end{pmatrix} , \sigma_z=\begin{pmatrix} 1 & 0 \\ 0 & -1 \end{pmatrix} ,
\end{eqnarray}
Dengan demikian spin adalah generator grup $SU(2)$. Identitas dari operator spin dapat terlihat dibawah ini.
\begin{eqnarray}
\sigma_i^2&=&\textbf{1}\rightarrow\sigma_x\sigma_x=\sigma_y\sigma_y=\sigma_z\sigma_z , \\
\sigma_i\sigma_j&=&-\sigma_j\sigma_i , \\
\sigma_j\sigma_k&=&i\epsilon_{jkl}\sigma_l , \\
\sigma_i\sigma_j+\sigma_j\sigma_i&=&0 ,
\end{eqnarray}
\begin{eqnarray}
[\sigma_j,\sigma_k]&=&2i\epsilon_{ijk}\sigma_l , \\
\{\sigma_i,\sigma_j\}&=&2\delta_{ij} ,
\end{eqnarray}
Untuk identitas komutator operator spin $(\textbf{S}=\frac{\hbar}{2}\boldsymbol{\sigma})$ memiliki kesamaan pada komutator kecepatan angular dalam mekanika kuantum.
\begin{eqnarray}
[S_j,S_k]=i\hbar\epsilon_{jkl}S_l\Leftrightarrow [J_j,J_k]=i\hbar\epsilon_{jkl}J_l ,
\end{eqnarray}

\chapter{Bogomolnyi-Prasad-Sommerfield (BPS)}
Dalam menyederhanakan kasus yang berkaitan dengan topologi ketika berhadapan dengan sebuah solusi pada persamaan nonlinier, biasanya digunakan persamaan Bogomolnyi. Persamaan Bogomolnyi adalah persamaan diferensial orde satu yang tereduksi dari persamaan diferensial orde dua atau lebih. Hal tersebut dilakukan guna mencari solusi analitik dari soliton, yang biasanya terdiri dari persamaan diferensial orde dua atau lebih. Persamaan Bogomolnyi tidak bergantung dengan waktu, sehingga solusi yang dihasilkan dari persamaan ini merupakan solusi energi statik paling stabil yang seminimal mungkin. Solusi persamaan Bogomolnyi pada energi statik stabil yang seminimal mungkin disebut dengan keadaan solusi BPS (Bogomolnyi-Prasad-Sommerfield)\cite{Manton}.

Contoh pada kasus \textit{kinks/Domain Walls} pada dimensi $(1+1)$, diberikan:
\begin{eqnarray}
\mathcal{L}=\frac{1}{2}\partial_\mu\phi\partial^\mu\phi-V(\phi) \label{eq:kink} ,
\end{eqnarray}
Dengan potensial:
\begin{eqnarray}
V(\phi)=\frac{\lambda}{4}(\phi^2-\nu^2)^2 \label{eq:potentialkink} ,
\end{eqnarray}
Berdasarkan persamaan euler-lagrange, diperoleh:
\begin{eqnarray}
-\frac{\partial V}{\partial\phi}-\partial_\mu\partial^\mu\phi=0 \label{eq:eulerlagrangepotential} ,
\end{eqnarray}
Persamaan (\ref{eq:eulerlagrangepotential}) berada dalam keadaaan statik ketika:
\begin{eqnarray}
\phi''=\frac{d^2\phi}{dx^2}=\frac{\partial V}{\partial\phi} \label{eq:secondderivation1} ,
\end{eqnarray}
Potensial (\ref{eq:potentialkink}) diturunkan terhadap $\phi$ diperoleh:
\begin{eqnarray}
\frac{\partial V}{\partial\phi}=\lambda\phi(\phi^2-\nu^2) \label{eq:derivation1} ,
\end{eqnarray}
Dengan memasukkan persamaan (\ref{eq:derivation1}) pada (\ref{eq:secondderivation1}) dapat ditulis:
\begin{eqnarray}
\phi''-\lambda\phi(\phi^2-\nu^2)=0 \label{eq:eulerlagrange2} ,
\end{eqnarray}
Persamaan (\ref{eq:kink}) divariasikan terhadap tensor metrik:
\begin{eqnarray}
\mathcal{L}&=&\frac{1}{2}g^{\mu\nu}\partial_\mu\phi\partial_\nu\phi-V(\phi) \nonumber \\
\delta_g\mathcal{L}&=&\frac{1}{2}(\delta_gg^{\mu\nu})\partial_\mu\phi\partial_\nu\phi \nonumber \\
\delta_g\mathcal{L}&=&-\frac{1}{2}g^{\mu\rho}(\delta g_{\rho\sigma})g^{\sigma\nu}\partial_\mu\phi\partial_\nu\phi \label{eq:variation1} ,
\end{eqnarray}
Dari (\ref{eq:variation1}) berdasarkan pendekatan teori relativitas umum didefinisikan:
\begin{eqnarray}
T^{\mu\nu}=-\mathcal{L}+\partial_\mu\phi\partial_\nu\phi ,
\end{eqnarray}
Kemudian dicari nilai $T^{00}$ yang didefinisikan sebagai densitas energi:
\begin{eqnarray}
\mathcal{E}=T^{00}=-\mathcal{L}=\frac{1}{2}\phi'^2+V \label{eq:densityenergy} ,
\end{eqnarray}
Dimana $\phi'^2=\partial_\mu\phi\partial_\nu\phi$. Dari (\ref{eq:densityenergy}), kita dapat menghitung energi:
\begin{eqnarray}
E&=&\int dx T^{00} \nonumber \\
&=&\frac{1}{2}\int dx[\phi'^2+2V] \nonumber \\
&=&\frac{1}{2}\int dx[(\phi'-\sqrt{2V})^2+\sqrt{2V}\phi'] \label{eq:energy1} ,
\end{eqnarray}
Persamaan (\ref{eq:energy1}) memiliki suku yang terdiri dari kuadrat orde dua, dimana nilai kuadrat orde dua selalu positif. Karena alasan tersebut dengan $(\phi'-\sqrt{2})^2\geq 0$ maka (\ref{eq:energy1}) dapat ditulis:
\begin{eqnarray}
E&\geq&\frac{1}{2}\int dx\phi'\sqrt{2V} \nonumber \\
E&\geq& E_{min} \label{eq:energy2} ,
\end{eqnarray}
Berdasarkan perhitungan energi pada (\ref{eq:energy2}), persamaan (\ref{eq:kink}) memiliki solusi BPS ketika:
\begin{eqnarray}
\phi'=\sqrt{2V} \label{eq:BPS} ,
\end{eqnarray}
Persamaan (\ref{eq:BPS}) diturunkan terhadap $x$:
\begin{eqnarray}
\phi'^2&=&2V \nonumber \\
\frac{d}{dx}(\phi'^2)&=&\frac{d}{dx}[2V(\phi(x))] \nonumber \\
2\phi'\phi''&=&2\phi'\frac{dV}{d\phi} \nonumber \\
\phi''&=&\frac{dV}{d\phi} ,
\end{eqnarray}
Terlihat bahwa $\phi'$ yang diturunkan terhadap $x$ akan kembali ke kondisi (\ref{eq:eulerlagrange2}), sehingga terbukti persamaan (\ref{eq:kink}) memiliki solusi persamaan orde dua yang sama dengan solusi persamaan orde satu, dapat dikatakan bahwa persamaan (\ref{eq:kink}) memiliki solusi BPS.

Pada kasus Skyrmion, tinjau persamaan (\ref{eq:staticenergy}) pada Lagrangian asli Skyrmion dengan menulis ulang suku integran kuadrat menjadi:
\begin{eqnarray}
E_{statik}=\frac{1}{12\pi^2}\int d^3x\left(-\frac{1}{2}\right)\operatorname{Tr}\left[\left(R_a\pm\frac{1}{4}\epsilon_{abc}[R_b,R_c]\right)^2\right] \label{eq:squaredintegran} ,
\end{eqnarray}
Dengan $a,b,c$ adalah indeks boneka. Dari persamaan (\ref{eq:squaredintegran}) diketahui:
\begin{eqnarray}
\left(R_a\pm\frac{1}{4}\epsilon_{abc}[R_b,R_c]\right)^2=R_a^2\pm\frac{1}{2}R_a\epsilon_{abc}[R_b,R_c]+\frac{1}{16}(\epsilon_{abc}[R_b,R_c])^2 \label{squaredterm1} ,
\end{eqnarray}
Dan pada (\ref{squaredterm1}):
\begin{eqnarray}
\epsilon_{abc}[R_b,R_c]&=&\epsilon_{abc}[R_bR_c-R_cR_b] \nonumber \\
&=&\epsilon_{abc}R_bR_c-\epsilon_{abc}R_cR_b \nonumber \\
&=&\epsilon_{abc}R_bR_c-\epsilon_{acb}R_bR_c \nonumber \\
&=&\epsilon_{abc}R_bR_c-(-\epsilon_{abc})R_bR_c \nonumber \\
&=&\epsilon_{abc}R_bR_c+\epsilon_{abc}R_bR_c \nonumber \\
&=&2\epsilon_{abc}R_bR_c \label{levicivita1} ,
\end{eqnarray}
Dimana $\epsilon_{abc}=-\epsilon_{acb}$, kemudian kita peroleh:
\begin{eqnarray}
(\epsilon_{abc}[R_b,R_c])^2=\epsilon_{abc}[R_b,R_c]\epsilon_{ade}[R_d,R_e]=2\epsilon_{abc}R_bR_c2\epsilon_{ade}R_dR_e \label{levicivita2} ,
\end{eqnarray}
Masukkan (\ref{levicivita1}) dan (\ref{levicivita2}) kedalam (\ref{squaredterm1}):
\begin{eqnarray}
\left(R_a\pm\frac{1}{4}\epsilon_{abc}[R_b,R_c]\right)^2&=&R_a^2\pm\frac{1}{2}R_a2\epsilon_{abc}R_bR_c+\frac{1}{16}(2\epsilon_{abc}R_bR_c2\epsilon_{ade}R_dR_e) \nonumber \\
&=&R_a^2\pm R_a2\epsilon_{abc}R_bR_c+\frac{1}{4}(\epsilon_{abc}R_bR_c\epsilon_{ade}R_dR_e) \label{squaredterm2} ,
\end{eqnarray}
Dari (\ref{levicivita2}) diketahui:
\begin{eqnarray}
2\epsilon_{abc}R_bR_c2\epsilon_{ade}R_dR_e=\epsilon_{abc}[R_b,R_c]\epsilon_{ade}[R_d,R_e] \label{levicivita3} , \\
\epsilon_{abc}R_bR_c\epsilon_{ade}R_dR_e=\frac{1}{2}\epsilon_{abc}[R_b,R_c]\frac{1}{2}\epsilon_{ade}[R_d,R_e] \label{levicivita4} ,
\end{eqnarray}
Sehingga (\ref{squaredterm2}) menjadi:
\begin{eqnarray}
\left(R_a\pm\frac{1}{4}\epsilon_{abc}[R_b,R_c]\right)^2&=&R_a^2\pm R_a2\epsilon_{abc}R_bR_c+\frac{1}{4}\left(\frac{1}{2}\epsilon_{abc}[R_b,R_c]\frac{1}{2}\epsilon_{ade}[R_d,R_e]\right) \nonumber \\
&=&R_a^2\pm R_a2\epsilon_{abc}R_bR_c+\frac{1}{4}\left(\frac{1}{4}\epsilon_{abc}\epsilon_{ade}[R_b,R_c][R_d,R_e]\right) \nonumber \\
&=&R_a^2\pm R_a2\epsilon_{abc}R_bR_c+\frac{1}{16}([\delta_{bd}\delta_{ce}-\delta_{be}\delta_{cd}][R_b,R_c][R_d,R_e]) \nonumber \\
&=&R_a^2\pm R_a2\epsilon_{abc}R_bR_c+\frac{1}{16}(\delta_{bd}\delta_{ce}[R_b,R_c][R_d,R_e] \nonumber \\
&&-\delta_{be}\delta_{cd}[R_b,R_c][R_d,R_e]) \nonumber \\
&=&R_a^2\pm R_a2\epsilon_{abc}R_bR_c+\frac{1}{16}([R_b,R_c][R_b,R_c]-[R_b,R_c][R_c,R_b]) \nonumber \\
&=&R_a^2\pm R_a2\epsilon_{abc}R_bR_c+\frac{1}{16}([R_b,R_c][R_b,R_c]+[R_b,R_c][R_b,R_c]) \nonumber \\
&=&R_a^2\pm R_a2\epsilon_{abc}R_bR_c+\frac{1}{16}(2[R_b,R_c][R_b,R_c]) \nonumber \\
&=&R_a^2\pm R_a2\epsilon_{abc}R_bR_c+\frac{1}{8}([R_b,R_c][R_b,R_c]) \label{squaredterm3} ,
\end{eqnarray}
Dengan memasukkan (\ref{squaredterm3}) ke persamaan (\ref{eq:squaredintegran}) diperoleh:
\begin{eqnarray}
E_{statik}&=&\frac{1}{12\pi^2}\int d^3x\left(-\frac{1}{2}\right)\operatorname{Tr}\left[\left(R_a^2\pm R_a\epsilon_{abc}R_bR_c+\frac{1}{8}\epsilon_{abc}[R_b,R_c]^2\right)\right] \nonumber \\
&=&\frac{1}{12\pi^2}\int d^3x\left(-\frac{1}{2}\right)\operatorname{Tr}(R_a^2)\pm\frac{1}{12\pi^2}\int d^3x\left(-\frac{1}{2}\right)\epsilon_{abc}\operatorname{Tr}(R_aR_bR_c) \nonumber \\
&&+\frac{1}{12\pi^2}\int d^3x\left(-\frac{1}{2}\right)\operatorname{Tr}\left(\frac{1}{8}\epsilon_{abc}[R_b,R_c]^2\right) \nonumber \\
&=&\frac{1}{12\pi^2}\int d^3x\left(-\frac{1}{2}\right)\operatorname{Tr}\left(R_a^2+\frac{1}{8}\epsilon_{abc}[R_b,R_c]^2\right) \nonumber \\
&&\pm\frac{1}{12\pi^2}\int d^3x\left(-\frac{1}{2}\right)\epsilon_{abc}\operatorname{Tr}(R_aR_bR_c) \label{eq:staticsquaredintegran1} ,
\end{eqnarray}
Agar persamaan (\ref{eq:staticsquaredintegran1}) sama dengan persamaan (\ref{eq:scalestatic}), maka perlu ditambahkan "sesuatu" suku.
\begin{eqnarray}
E_{statik}&=&\frac{1}{12\pi^2}\int d^3x\left(-\frac{1}{2}\right)\operatorname{Tr}\left(R_a^2+\frac{1}{8}\epsilon_{abc}[R_b,R_c]^2\right) \nonumber \\
&&\pm\frac{1}{12\pi^2}\int d^3x\left(-\frac{1}{2}\right)\epsilon_{abc}\operatorname{Tr}(R_aR_bR_c) \nonumber \\
&&\mp\frac{1}{12\pi^2}\int d^3x\left(-\frac{1}{2}\right)\epsilon_{abc}\operatorname{Tr}(R_aR_bR_c) \label{eq:staticsquaredintegran2} ,
\end{eqnarray}
Sehingga persamaan (\ref{eq:scalestatic}) dapat ditulis:
\begin{eqnarray}
E_{statik}&=&\frac{1}{12\pi^2}\int d^3x\left(-\frac{1}{2}\right)\operatorname{Tr}\left[\left(R_a\pm\frac{1}{4}\epsilon_{abc}[R_b,R_c]\right)^2\right. \nonumber \\
&&\left.\pm\epsilon_{abc}\operatorname{Tr}(R_aR_bR_c)\right] \label{eq:staticsquaredintegran3} , \\
E_{statik}&=&\frac{1}{12\pi^2}\int d^3x\left(-\frac{1}{2}\right)\operatorname{Tr}\left[\left(R_a\pm\frac{1}{4}\epsilon_{abc}[R_b,R_c]\right)^2\right] \nonumber \\
&&\mp\frac{1}{24\pi^2}\int d^3x\epsilon_{abc}\operatorname{Tr}(R_aR_bR_c) \label{eq:staticsquaredintegran4} ,
\end{eqnarray}

Sekarang tinjau persamaan (\ref{eq:staticsquaredintegran4}), untuk memperoleh persamaan Bogomolnyi, suku integran kuadrat pada persamaan dianggap nol, sehingga:
\begin{eqnarray}
E_{statik}\geq\mp\frac{1}{24\pi^2}\int d^3x\epsilon_{abc}\operatorname{Tr}(R_aR_bR_c) \label{eq:staticsquaredintegran5} ,
\end{eqnarray}
Agar memiliki nilai positif pada integral pertama maka:
\begin{eqnarray}
E_{statik}\geq-\frac{1}{24\pi^2}\int d^3x\epsilon_{abc}\operatorname{Tr}(R_aR_bR_c) \label{eq:staticsquaredintegran6} ,
\end{eqnarray}
Integral pada persamaan (\ref{eq:staticsquaredintegran6}) didefinisikan sebagai:
\begin{eqnarray}
B=-\frac{1}{24\pi^2}\int d^3x\epsilon_{abc}\operatorname{Tr}(R_aR_bR_c) \label{eq:topologicalcharge} ,
\end{eqnarray}
Dimana $B$ merupakan disebut sebagai muatan topologi. Hasil perhitungan diatas dapat ditulis:
\begin{eqnarray}
E_{statik}\geq B
\end{eqnarray}
Hal tersebut dapat menunjukkan bahwa Skyrmion tidak memiliki solusi BPS.

\chapter{Teorema Derrick}
Pada setiap teori medan, fungsi energi untuk medan statik yang divariasikan terhadap transformasi ruang tidak pernah bernilai nol untuk tiap konfigurasi medan non-vakum. Hal tersebut mengindikasikan bahwa tidak ada solusi energi statik berhingga kecuali pada daerah vakum. Solusi soliton yang stabil mensyaratkan bahwa titik stasioner pada konfigurasi medan untuk semua variasi yang bertransformasi skala haruslah sama. Solusi soliton yang stabil memiliki kondisi transformasi skala yang stasioner pada daerah tidak vakum\cite{Manton}.

Tinjau kembali persamaan Lagrangian \textit{kink} pada energi $n$ spasial:
\begin{eqnarray}
E&=&\int d^nx\left[\frac{1}{2}(\nabla\phi)^2+V\right] \nonumber \\
&=&E_2+E_0 \label{eq:energy3} ,
\end{eqnarray}
Misalkan $x$ bertransformasi skala menjadi:
\begin{eqnarray}
\mathbf{x}\rightarrow\mathbf{x}'&\equiv&\lambda\mathbf{x} , \nonumber \\
\nabla\phi\rightarrow\nabla\phi(\mathbf{x}')&\equiv&\lambda\nabla\phi(\mathbf{x}) ,
\end{eqnarray}
Sehingga persamaan (\ref{eq:energy3}) menjadi:
\begin{eqnarray}
E&=&\int d^nx'\left[\lambda^{-n}\left(\frac{1}{2}\lambda^2(\nabla\phi)^2+V\right)\right] , \\
E^{(\lambda)}&=&E_2\lambda^{2-n}+E_0\lambda^{-n} \label{eq:energy4} ,
\end{eqnarray}
Dari persamaan (\ref{eq:energy4}) diperoleh:
\begin{eqnarray}
\frac{dE}{d\lambda}&=&(2-n)E_2\lambda^{1-n}-nE_0\lambda^{-n-1} \label{eq:derrick1} , \\
\frac{d^2E}{d\lambda^2}&=&(2-n)(1-n)E_2\lambda^{-n}+n(n+1)E_0\lambda^{-n-2} \label{eq:derrick2} ,
\end{eqnarray}
Persamaan (\ref{eq:energy4}) keadaannya akan stabil ketika memenuhi syarat kestabilan minimum:
\begin{eqnarray}
\frac{dE}{d\lambda}\bigg|_{\lambda=1}=0 \label{eq:condition1} , \\
\frac{d^2E}{d\lambda^2}\bigg|_{\lambda=1}>0 \label{eq:condition2} ,
\end{eqnarray}
Dari kondisi (\ref{eq:condition1}) dan (\ref{eq:condition2}), didapatkan bahwa:
\begin{eqnarray}
\frac{dE}{d\lambda}\bigg|_{\lambda=1}&=&(2-n)E_2-nE_0 \nonumber \\
(2-n)E_2-nE_0&=&0 \nonumber \\
(2-n)E_2&=&nE_0 ,
\end{eqnarray}
Dan:
\begin{eqnarray}
\frac{d^2E}{d\lambda^2}\bigg|_{\lambda=1}&=&(2-n)(1-n)E_2+n(n+1)E_0 \nonumber \\
(2-n)(1-n)E_2+n(n+1)E_0&>&0 ,
\end{eqnarray}
Kondisi (\ref{eq:condition2}) akan terpenuhi pada $n=1$.

Pada kasus Skyrmion kita tinjau persamaan Lagrangian asli Skyrmion (\ref{eq:staticenergy}) dengan mengabaikan suku potensial, kemudian dilakukan transformasi skala:
\begin{eqnarray}
x^a=\frac{2\tilde{x}^a}{ef_\pi}\rightarrow dx^a=\left(\frac{2}{ef_\pi}\right)d\tilde{x}^a \label{scale} ,
\end{eqnarray}
Untuk indeks $a$ berjalan pada $1,2,3$, dan berlaku pula pada indeks $b$ dan $c$, sehingga diperoleh:
\begin{eqnarray}
d^3x=dx^1dx^2dx^3=\left(\frac{2d\tilde{x}^1}{ef_\pi}\right)\left(\frac{2d\tilde{x}^2}{ef_\pi}\right)\left(\frac{2d\tilde{x}^3}{ef_\pi}\right)=\left(\frac{2}{ef_\pi}\right)^3d\tilde{x}^1d\tilde{x}^2d\tilde{x}^3 \label{eq:spatial} ,
\end{eqnarray}

Sekarang kita tinjau:
\begin{eqnarray}
R_a=U^\dagger\partial_aU=U^\dagger\frac{\partial U}{\partial x^a} \label{eq:chiral} ,
\end{eqnarray}
Persamaan (\ref{eq:chiral}) dapat dinyatakan dengan:
\begin{eqnarray}
R_a=U^\dagger\frac{\partial\tilde{x}^b}{\partial x^a}\frac{\partial U}{\partial\tilde{x}^b} \label{eq:scalechiral} ,
\end{eqnarray}
Dari persamaan (\ref{eq:scalechiral}) dan (\ref{scale}), diperoleh:
\begin{eqnarray}
\frac{\partial\tilde{x}^b}{\partial x^a}=\frac{(\frac{af_\pi}{2})\partial x^b}{\partial x^a}=\left(\frac{af_\pi}{2}\right)\frac{\partial x^b}{\partial x^a}=\left(\frac{af_\pi}{2}\right)\delta_a^b \label{eq:scalechainrule} ,
\end{eqnarray}
Dengan $\frac{\partial x^b}{\partial x^a}=\delta_a^b$, dan $delta_a^b=1$ untuk $a=b$. Kemudian substitusi (\ref{eq:scalechainrule}) ke persamaan (\ref{eq:scalechiral}), maka diperoleh:
\begin{eqnarray}
R_a=U^\dagger\frac{\partial\tilde{x}^b}{\partial x^a}\frac{\partial U}{\partial\tilde{x}^b}=U^\dagger\left(\frac{ef_\pi}{2}\right)\delta_a^b\frac{\partial U}{\partial\tilde{x}^b}=\left(\frac{ef_\pi}{2}\right)U^\dagger\frac{\partial U}{\partial\tilde{x}^a}=\left(\frac{ef_\pi}{2}\right)\tilde{R}_a \label{eq:chiralscale} ,
\end{eqnarray}

Kita tinjau transformasi skala pada suku pertama persamaan (\ref{eq:staticenergy}).
\begin{eqnarray}
E_{statik}^{(1)}=-\int d^3x\operatorname{Tr}\left[\frac{f_\pi^2}{16}R_a^2\right] \label{eq:term1} ,
\end{eqnarray}
Substitusikan (\ref{eq:spatial}) dan (\ref{eq:chiralscale}) ke persamaan (eq:term1):
\begin{eqnarray}
E_{statik}^{(1)}&=&-\int d^3x\operatorname{Tr}\left[\frac{f_\pi^2}{16}R_a^2\right]  \nonumber \\
&=&-\left(\frac{2}{af_\pi}\right)^3\left(\frac{f_\pi^2}{16}\right)\left(\frac{ef_\pi}{2}\right)^2\int d\tilde{x}^1d\tilde{x}^2d\tilde{x}^3\operatorname{Tr}[\tilde{R}_a^2] \nonumber \\
&=&-\left(\frac{8}{e^3f_\pi^3}\right)\left(\frac{f_\pi^2}{16}\right)\left(\frac{e^2f_\pi^2}{4}\right)\int d\tilde{x}^1d\tilde{x}^2d\tilde{x}^3\operatorname{Tr}[\tilde{R}_a^2] \nonumber \\
&=&\frac{1}{12\pi^2}\int d^3x\left(-\frac{1}{2}\right)\operatorname{Tr}[R_a^2] \label{eq:term1a} ,
\end{eqnarray}
Dimana telah dilabel ulangkan $\tilde{x}\rightarrow x$ dan $\tilde{R}_a\rightarrow R_a$, dan kita gunakan definisi:
\begin{eqnarray}
-\left(\frac{f_\pi}{8e}\right)=-\frac{1}{2}\left(\frac{f_\pi}{4e}\right)=-\frac{1}{2}\left(\frac{1}{12\pi^2}\right) ,
\end{eqnarray}

Tinjau transformasi skala untuk suku kedua persamaan (\ref{eq:staticenergy}):
\begin{eqnarray}
E_{statik}^{(2)}&=&-\int d^3x\operatorname{Tr}\left(\frac{1}{32e^2}[R_a,R_c]^2\right) \nonumber \\
&=&-\int d^3x\operatorname{Tr}[R_aR_c-R_cR_a]^2 \nonumber \\
&=&-\int \left(\frac{2}{ef_\pi}\right)^3d\tilde{x}^1d\tilde{x}^2d\tilde{x}^3\operatorname{Tr}\left(\frac{1}{32e^2}\left[\left(\frac{ef_\pi}{2}\right)\tilde{R}_a\left(\frac{ef_\pi}{2}\right)\tilde{R}_c\right.\right. \nonumber \\
&&-\left.\left.\left(\frac{ef_\pi}{2}\right)\tilde{R}_c\left(\frac{ef_\pi}{2}\right)\tilde{R}_a\right]^2\right) \nonumber \\
&=&-\left(\frac{2}{ef_\pi}\right)^3\left(\frac{1}{32e^2}\right)\int d\tilde{x}^1d\tilde{x}^2d\tilde{x}^3\operatorname{Tr}\left(\left(\frac{ef_\pi}{2}\right)^2\tilde{R}_a^2\left(\frac{ef_\pi}{2}\right)^2\tilde{R}_c^2\right. \nonumber \\
&&\left.-2\left(\frac{ef_\pi}{2}\right)\tilde{R}_a\left(\frac{ef_\pi}{2}\right)\tilde{R}_c\left(\frac{ef_\pi}{2}\right)\tilde{R}_c\left(\frac{ef_\pi}{2}\right)\tilde{R}_a+\left(\frac{ef_\pi}{2}\right)^2\tilde{R}_c^2\left(\frac{ef_\pi}{2}\right)^2\tilde{R}_a^2\right) \nonumber \\
&=&-\left(\frac{2}{ef_\pi}\right)^3\left(\frac{1}{32e^2}\right)\left(\frac{ef_\pi}{2}\right)^4\int d\tilde{x}^1d\tilde{x}^2d\tilde{x}^3\operatorname{Tr}[\tilde{R}_a^2\tilde{R}_c^2-2\tilde{R}_a\tilde{R}_c\tilde{R}_c\tilde{R}_a+\tilde{R}_c^2\tilde{R}_a^2] \nonumber \\
&=&-\left(\frac{8}{e^3f_\pi^3}\right)\left(\frac{1}{32e^2}\right)\left(\frac{e^4f_\pi^4}{16}\right)\int d^3\tilde{x}\operatorname{Tr}([\tilde{R}_a,\tilde{R}_c]^2) \nonumber \\
&=&-\left(\frac{f_\pi}{64e}\right)\int d^3\tilde{x}\operatorname{Tr}([\tilde{R}_a,\tilde{R}_c]^2) \nonumber \\
&=&\frac{1}{12\pi^2}\int d^3\tilde{x}\left(-\frac{1}{2}\right)\operatorname{Tr}\left(\frac{1}{8}[\tilde{R}_a,\tilde{R}_c]^2\right) ,
\end{eqnarray}
Dimana
\begin{eqnarray}
-\frac{f_\pi}{64e}=\frac{1}{8}\left(-\frac{1}{2}\right)\left(\frac{f_\pi}{4e}\right)=\frac{1}{8}\left(-\frac{1}{2}\right)\left(\frac{1}{12\pi^2}\right) ,
\end{eqnarray}
Dan telah dilabel ulangkan $\tilde{x}\rightarrow x$ dan $\tilde{R}_a\rightarrow R_a$, sehingga:
\begin{eqnarray}
E_{statik}^{(2)}=\frac{1}{12\pi^2}\int d^3x\left(-\frac{1}{2}\right)\operatorname{Tr}\left(\frac{1}{8}[R_a,R_c]^2\right) \label{eq:term2} ,
\end{eqnarray}

Dari persamaan (\ref{eq:term1a}) dan (\ref{eq:term2}) diperoleh:
\begin{eqnarray}
E_{statik}&=&E_{statik}^{(1)}+E_{statik}^{(2)} \nonumber \\
&=&\frac{1}{12\pi^2}\int d^3x\left(-\frac{1}{2}\right)\operatorname{Tr}[R_a^2]+\frac{1}{12\pi^2}\int d^3x\left(-\frac{1}{2}\right)\operatorname{Tr}\left(\frac{1}{8}[R_a,R_c]^2\right) \nonumber \\
&=&\frac{1}{12\pi^2}\int d^3x\left(-\frac{1}{2}\right)\operatorname{Tr}\left[R_a^2+\frac{1}{8}[R_a,R_c]^2\right] \label{eq:scalestatic} ,
\end{eqnarray}

Persamaan (\ref{eq:scalestatic}) dilakukan pengujian kestabilan skala terhadap transformasi skala $x\rightarrow\lambda x$, sehingga $R_a$ terskalakan sebagai berikut:
\begin{eqnarray}
R_a(x)&\rightarrow& U^\dagger(x)\frac{\partial U(x)}{\partial x^a} \nonumber \\
&=&U^\dagger(\lambda x)\frac{\partial U(\lambda x)}{\partial x^a} \nonumber \\
&=&\lambda U^\dagger(\lambda x)\frac{\partial U(\lambda x)}{\partial\lambda x^a} \nonumber \\
&=&\lambda R_a(\lambda x) \label{chiraltransform} ,
\end{eqnarray}
Dengan memasukkan (\ref{chiraltransform}) ke persamaan (\ref{eq:scalestatic}) maka:
\begin{eqnarray}
E[(\lambda)]_{statik}&=&\frac{1}{12\pi^2}\int \frac{d^3(\lambda x)}{\lambda^3}\left(-\frac{1}{2}\right)\operatorname{Tr}\left[(\lambda R_a(\lambda x))^2+\frac{1}{8}([\lambda R_a(\lambda x),\lambda R_c(\lambda x)]^2)\right] \nonumber \\
&=&\frac{1}{12\pi^2}\int \frac{d^3(\lambda x)}{\lambda^3}\left(-\frac{1}{2}\right)\operatorname{Tr}\left[\lambda^2 R_a^2+\frac{1}{8}([\lambda R_a\lambda R_c-\lambda R_c\lambda R_a]^2)\right] \nonumber \\
&=&\frac{1}{12\pi^2}\int \frac{d^3(\lambda x)}{\lambda^3}\left(-\frac{1}{2}\right)\operatorname{Tr}\left[\lambda^2 R_a^2+\frac{1}{8}([\lambda R_a\lambda R_c\lambda R_a\lambda R_c-2\lambda R_a\lambda R_c\lambda R_c\lambda R_a \right. \nonumber \\
&&\left.+\lambda R_c\lambda R_a\lambda R_c\lambda R_a])\right] \nonumber \\
&=&\frac{1}{12\pi^2}\int \frac{d^3(\lambda x)}{\lambda^3}\left(-\frac{1}{2}\right)\operatorname{Tr}\left[\lambda^2 R_a^2+\frac{\lambda^4}{8}([R_aR_cR_aR_c-2R_aR_cR_cR_a \right. \nonumber \\
&&\left.+R_cR_aR_cR_a])\right] \nonumber \\
&=&\frac{1}{12\pi^2}\int \frac{d^3(\lambda x)}{\lambda^3}\left(-\frac{1}{2}\right)\operatorname{Tr}\left[\lambda^2 R_a^2+\frac{\lambda^4}{8}[R_aR_c-R_cR_a]^2\right] \nonumber \\
&=&\frac{1}{12\pi^2}\int \frac{d^3(\lambda x)}{\lambda^3}\left(-\frac{1}{2}\right)\operatorname{Tr}\left[\lambda^2 R_a^2+\frac{\lambda^4}{8}[R_a,R_c]^2\right] \nonumber \\
&=&\frac{1}{12\pi^2}\int \frac{d^3(\lambda x)}{\lambda^3}\left(-\frac{1}{2}\right)\operatorname{Tr}[\lambda^2 R_a^2]+\frac{1}{12\pi^2}\int \frac{d^3(\lambda x)}{\lambda^3}(-\frac{1}{2})\operatorname{Tr}\left[\frac{\lambda^4}{8}[R_a,R_c]^2\right] \nonumber \\
&=&\frac{1}{\lambda}\left(\frac{1}{12\pi^2}\int \frac{d^3(\lambda x)}{\lambda^3}\left(-\frac{1}{2}\right)\operatorname{Tr}[\lambda^2 R_a^2]\right) \nonumber \\
&&+\lambda\left(\frac{1}{12\pi^2}\int \frac{d^3(\lambda x)}{\lambda^3}\left(-\frac{1}{2}\right)\operatorname{Tr}\left[\frac{\lambda^4}{8}[R_a,R_c]^2\right]\right) \nonumber \\
&=&\frac{1}{\lambda}E_\sigma+\lambda E_{sky} \label{eq:staticscale} ,
\end{eqnarray}
Dimana $E_\sigma$ adalah suku energi \textit{chiral} statik dan $E_{sky}$ adalah suku energi Skyrme. Berdasarkan teorema Derrick yang menyatakan bahwa energi akan stabil pada kondisi:
\begin{eqnarray}
\frac{dE[\lambda]}{d\lambda}\bigg|_{\lambda=1}=0 \label{condition1} \\
\frac{d^2E[\lambda]}{d\lambda^2}\bigg|_{\lambda=1}>0 \label{condition2} ,
\end{eqnarray}
Sehingga dari (\ref{eq:staticscale}) dihitung:
\begin{eqnarray}
\frac{dE[\lambda]}{d\lambda}\bigg|_{\lambda=1}&=&\frac{d}{d\lambda}\left[\frac{1}{\lambda}E_\sigma+\lambda E_{sky}\right]\bigg|_{\lambda=1} \nonumber \\
&=&\left[\left(\frac{d}{d\lambda}\left[\frac{1}{\lambda}\right]\right)E_\sigma+\frac{1}{\lambda}\left(\frac{d}{d\lambda}E_\sigma\right)+\left(\frac{d}{d\lambda}[\lambda]\right) E_{sky}+\lambda\left(\frac{d}{d\lambda}E_{sky}\right)\right]\bigg|_{\lambda=1} \nonumber \\
&=&\left[-\frac{1}{\lambda^2}E_\sigma+E_{sky}\right]\bigg|_{\lambda=1} \nonumber \\
&=&-E_\sigma+E_{sky} \label{eq:derivative1} ,
\end{eqnarray}
\begin{eqnarray}
\frac{d^2E[\lambda]}{d\lambda^2}\bigg|_{\lambda=1}&=&\frac{d}{d\lambda}\left[-\frac{1}{\lambda^2}E_\sigma+E_{sky}\right]\bigg|_{\lambda=1} \nonumber \\
&=&(-1)\frac{d}{d\lambda}\left[\frac{1}{\lambda^2}E_\sigma\right]\bigg|_{\lambda=1}+\frac{d}{d\lambda}E_{sky}\bigg|_{\lambda=1} \nonumber \\
&=&(-1)\left[\left(\frac{d}{d\lambda}\left[\frac{1}{\lambda^2}E_\sigma\right]\right)E_\sigma+\frac{1}{\lambda^2}\left(\frac{d}{d\lambda}\right)E_\sigma\right]\bigg|_{\lambda=1}+0 \nonumber \\
&=&(-1)\left[-\frac{2}{\lambda^3}E_\sigma+0\right]\bigg|_{\lambda=1} \nonumber \\
&=&\frac{2}{\lambda^3}E_\sigma\bigg|_{\lambda=1} \nonumber \\
&=&2E_\sigma \label{eq:derivative2} ,
\end{eqnarray}
Agar kondisi (\ref{condition1}) terpenuhi, maka $E_\sigma=E_{sky}$, dengan $E_\sigma\geq 0$ untuk kondisi (\ref{condition2}).												%
%																%
\end{document}													%
%																%
% ++++++++++++++++++++++++++++++++++++++++++++++++++++++++++++++%
%					 ~~~~~ SELESAI ~~~~~						%
% ++++++++++++++++++++++++++++++++++++++++++++++++++++++++++++++%