% +++++++++++++++++++++++++++
%
% BAB 5: Kesimpulan dan saran
%
% +++++++++++++++++++++++++++

\chapter{KESIMPULAN}

\section{Kesimpulan}
Hasil perhitungan energi statik pada menggunakan model Skyrme $SU(2)$ dengan Lagrangian suku $L_2$ dan $L_6$ memiliki solusi yang mendekati BPS, dimana solusi tersebut memiliki error yang melebihi sekitar $8\%$. Untuk fungsi profil yang menghasilkan solusi mendekati BPS tersebut, kita dapat menggunakannya dalam menghitung energi rotasi selanjutnya. 

Energi rotasi yang diperoleh memiliki bentuk yang sama dengan perhitungan sebelumnya yang menggunakan suku semua suku, dengan definisi momen inersia yang berbeda karena kehilangan suku $L_4$. Energi rotasi juga mendeskripsikan gerakan rotasi dan isorotasi pada \textit{rigid body} dari inti nuklir pada model Skyrme, walaupun nukleon yang dideskripsikan pada model ini belum kami pisahkan karakteristik dari proton dan neutron. 

Energi statik dan energi rotasi yang diperoleh dengan menggunakan solusi BPS dapat kita pelajari deskripsi energi ikat, dimana pada perhitungan sebelumnya yang menggunakan suku $L_6$ dan $L_0$ mendeskripsikan arti fisis pada inti nuklir untuk mempelajari energi ikat. Pengaruh dari energi statik lebih besar daripada energi rotasi pada model Skyrme, sehingga pernyataan tersebut sesuai dengan kondisi sistem \textit{rigid body} yang ada pada teori model nuklir.

Untuk nilai konstanta kopling $\lambda$ yang diperoleh, kita peroleh nilai terbaik dengan $\lambda=1,077\times 10^{-3}$ karena memiliki tingkat \textit{error} terendah pada perhitungan massa nukleon dan massa delta. Nilai konstanta kopling tersebut nantinya dibandingkan dengan nilai konstanta kopling pada perhitungan \textit{fitting} energi ikat model Skyrme.

\section{Saran}

Dari hasil yang disajikan dalam skripsi ini, studi energi ikat untuk model Skyrme masih sangatlah kurang, mengingat bahwa energi ikat pada inti nuklir tidak hanya terdiri dari energi statik dan energi rotasi, dan \textit{fitting} data untuk memperoleh parameter model tersebut belum dilakukan. Pengaruh dari energi coulomb dan energi isospin \textit{symmetry breaking} yang telah dihitung pada Lagrangian Skyrmion belum dilakukan. Bahkan energi tegangan permukaan pada model Skyrme belum pernah dibahas sebelumnya, sehingga penelitian lebih lanjut perlu dilakukan untuk mempelajari energi ikat yang lebih lanjut pada model Skyrme $SU(2)$.