% ==========================================================================	%
%																			%
%           		 					Abstrak									%
% 																			%
% ==========================================================================	%
%																			%
% SANGAT PENTING UNTUK DIEDIT												%
%																			%
% Pengaturan Halaman -------------------------------------------------------	%
%																			%
%Set nomor halaman															%
\setcounter{page}{8}															%
%Memasukkan ke daftar isi													%
\addcontentsline{toc}{chapter}{\numberline{}Abstrak}							%
%																			%
% ==========================================================================	%
% 							  ABSTRAK (INDONESIA)							%
% ==========================================================================	%
%																			%
%Judul Abstrak																%
\chapter*{ABSTRAK}															%
%																			%
%Identitas																	%
\begin{tabbing}																%
Nama		\hspace{1.5cm}	\= : \= \penulis\\									%
Program Studi 			\> : \> \program\\									%
Judul 		 			\> : \> Simulasi Monte Carlo untuk Verifikasi Hasil Pengukuran \\
                        \>   \> \textit{Quality Control} Sistem SPECT dengan Phantom Jaszsczak \\										%
\end{tabbing}																%
%																			%
%Isi Abstrak																	%
\noindent Citra kualitas tinggi diperlukan dalam memperoleh informasi yang akurat dalam menentukan posisi anatomi atau fungsional organ tubuh. Pada kedokteran nuklir, salah satu sistem pencitraan yang digunakan adalah SPECT (\textit{Single Photon Emission Computed Tomography}). Untuk menghasilkan citra kualitas tinggi diperlukan inspeksi dan validasi kuantitatif dari pengukuran sistem SPECT, sehingga QC (\textit{Quality Control}) sistem SPECT harus dilakukan untuk melihat performa dari pencitraan tersebut. Komponen seperti uniformitas, kontras dan resolusi spasial menjadi variabel penting dalam melihat hasil pengukuran dari sistem SPECT yang telah dilakukan. Dalam pengukuran, artefak dapat terjadi sehingga menurunkan kualitas citra yang dihasilkan.  Pada penelitian ini dilakukan ketiga pengukuran tersebut secara bersamaan dengan menggunakan fantom Jaszsczak. Pengukuran dilakukan secara manual berdasarkan protokol standar QC, dan sebagai justifikasi pengukuran manual akan dilakukan simulasi Monte Carlo sebagai data pembanding.
\vspace*{0.6cm}\\															%
%																			%
%Kata Kunci																	%
\noindent \textbf{Kata kunci:}\\												%
Citra, Fantom Jaszsczak, \textit{Quality Control}, Simulasi Monte Carlo, SPECT 																	%
%																			%
% ==========================================================================	%
% 								ABSTRAK (ENGLISH)							%
% ==========================================================================	%
%																			%
%Judul Abstrak																%
\chapter*{ABSTRACT}															%
%																			%
%Identitas																	%
\begin{tabbing}																%
Name		\hspace{2.2cm}	\= : \= \penulis\\									%
Program of Study 		\> : \> \programeng\\								%
Title 					\> : \> Monte Carlo Simulation for Verification Measurements  \\ 
						\>   \> Result of Quality Control SPECT System using Jaszsczak \\
						\>   \> Phantom \\										%
\end{tabbing}																%
%																			%
%Isi Abstrak																	%
\noindent High quality image is required in obtaining accurate information in determining the anatomical or functional position of organs. In nuclear medicine, one of the imaging systems used is SPECT (Single Photon Emission Computed Tomography). To produce high quality imagery required quantitative inspection and validation of SPECT system measurements, so SPECT system quality control (QC) should be performed to see the performance of the imaging. Components such as spatial resolution, contrast, and inhomogeneity are important variables in viewing the measurement results of SPECT systems that have been done. In the measurement, artifacts can occur thus decreasing the quality of the resulting image. In this study, these three measurements were performed simultaneously using Jaszsczak phantom. Measurements are done manually based on standard QC protocols, and as a justification for manual measurements Monte Carlo simulations will be performed as comparison data.
\vspace*{0.6cm}\\															%
%																			%
%Kata Kunci																	%
\noindent \textbf{Keywords:}\\												%
Image, Jaszczak Phantom, Quality Control, Monte Carlo Simulation, SPECT		%
%																			%
% ++++++++++++++++++++++++++++++++++++++++++++++++++++++++++++++++++++++++++	%
% 							~~~~~ Selesai ~~~~~								%
% ++++++++++++++++++++++++++++++++++++++++++++++++++++++++++++++++++++++++++	%