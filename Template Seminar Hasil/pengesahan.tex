% ======================================================================	%
%																		%
%           		 		    Halaman Pengesahan							%
% 																		%
% ======================================================================	%
%																		%
% TIDAK PERLU DIEDIT														%
%																		%
% Pengaturan Halaman ---------------------------------------------------	%
%																		%
%Set nomor halaman														%
\setcounter{page}{4}														%
%Memasukkan ke daftar isi												%
\addcontentsline{toc}{chapter}{\numberline{}Halaman Pengesahan}			%
%																		%
% Isi ------------------------------------------------------------------	%
%																		%
%Judul Halaman															%
\chapter*{HALAMAN PENGESAHAN}											%
%																		%
%Identitas																%
\begin{tabbing}															%
Skripsi ini diajukan oleh\\												%
Nama		\hspace{1.5cm}	\= : \= \penulis\\								%
NPM 						\> : \> \npm\\									%
Program Studi 			\> : \> \program\\								%
Judul Skripsi 			\> : \> Simulasi Monte Carlo untuk Verifikasi Hasil Pengukuran \\
                        \>   \> \textit{Quality Control} Sistem SPECT dengan Fantom Jaszsczak \\ 								%
\end{tabbing}															%
%																		%
%Pernyataan 																%
\textbf{Telah berhasil dipertahankan di hadapan Dewan Penguji dan 
	diterima sebagai bagian persyaratan yang diperlukan untuk memperoleh
	gelar \gelar~ Sains pada Program Studi \program, \fakultas,
	\universitas}														%
\vspace{0.2cm}															%
%																		%
%Pengesahan oleh Pembimbing & Penguji									%
\begin{center}															%
\bold{DEWAN PENGUJI}														%
\end{center}																%
\begin{tabular}{l l l l }												%
	& & & \\																%
	Pembimbing I&: & \pembimbingsatu & (\hspace*{3.0cm}) \\[1cm]			%
	& & & \\																%
	Pembimbing II&: &\pembimbingdua & (\hspace*{3.0cm}) \\[1cm]				%
	& & & \\																%
	%Penguji I&: &\pengujisatu & (\hspace*{3.0cm}) \\[1cm]				%
\end{tabular}															%
\vspace*{2cm}\\															%
%																		%
%Tempat & Tanggal Pengesahan												%
\begin{tabular}{l l l}													%
	Ditetapkan di	&: & Depok\\											%
	Tanggal			&: & \tanggallulus \\								%
\end{tabular}															%
%																		%
% ++++++++++++++++++++++++++++++++++++++++++++++++++++++++++++++++++++++	%
% 							~~~~~ Selesai ~~~~~							%
% ++++++++++++++++++++++++++++++++++++++++++++++++++++++++++++++++++++++	%