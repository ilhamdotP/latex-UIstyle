% +++++++++++++++++++++++++++
%
% BAB 3: Metodologi
%
% +++++++++++++++++++++++++++

\chapter{METODE PENELITIAN}

\section{Alat dan Bahan}
Pada penelitian ini menggunakan beberapa alat dan bahan sebagai berikut.

\begin{itemize}
\item SPECT/CT Siemens \\
SPECT/CT dengan varian Siemens Symbia T series Intevo 16 di rumah sakit kanker Dharmais akan digunakan sebagai alat pencitraan. Namun fokus dari penelitian ini hanya pada penggunaan SPECT, sehingga untuk penggunaan CT tidak dipakai. Sistem SPECT yang akan digunakan terdiri dari 2 kepala kamera gamma yang sudah dihubungkan dengan komputer untuk melakukan akuisisi data dan rekonstruksi citra.

\item Technetium-99m (\textsuperscript{99m}Tc) \\
Radionuklida \textsuperscript{99m}Tc memiliki waktu paruh selama 6 jam dengan energi gamma yang dipancarkan sekitar 140.51 keV. Karakteristik dari \textsuperscript{99m}Tc tidak memancarkan partikel beta, konstanta sinar gamma spesifik sekitar 0.076 mrem/jam pada 1 meter per 1 mCi, dan aktivitas spesifiknya adalah 5,243,820 Ci/gr untuk \textsuperscript{99m}Tc murni dan $3.4 \times 10^6$ Ci/gr untuk \textsuperscript{99m}Tc bentuk pertechnetate.
\item Phantom Jazsczak \\
Model phantom Jazszcak yang akan digunakan adalah model \textit{Deluxe Flangeless} standar yang terdiri dari diameter silinder terstandarisasi dan beberapa sisipan. Sisipan pada phantom terdiri dari enam bola padat, sisipan \textit{Cold Rods} dan kapiler sebagai tempat dalam meletakkan sumber radionuklida dalam phantom. Ukuran diameter \textit{Cold Rods} adalah; 4.8 mm, 6.4 mm, 7.9 mm, 9.5 mm, 11.1 mm, 12.7 mm, dan ukuran diameter bola padat adalah; 9.5 mm, 12.7 mm, 15.9 mm, 19.1 mm, 25.4 mm, 31.8 mm.

\begin{figure}[h]
\includegraphics[scale=0.05]{Gambar/jaszsczak.jpg}
\caption{Phantom Jaszczak}
\label{jaszsczak}
\end{figure}

\item Perangkat Lunak \\
Untuk melakukan simulasi Monte Carlo digunakan beberapa perangkat lunak seperti GATE (\textit{Geant4 Application for Tomographic Emission}). GATE dikembangkan dalam melakukan simulasi numerikal yang berkaitan dengan pencitraan medis dan radioterapi, terutama untuk simulasi emisi tomografi seperti PET, SPECT, CT. Hasil dari algoritma perangkat ini akan dibandingkan dengan hasil dari eksperimen yang dilakukan.
\end{itemize}

\section{Metode}
Penelitian ini menggunakan dua metode, yaitu eksperimen dan simulasi. Pengukuran eksperimen sistem SPECT dilakukan dengan menggunakan kolimasi LEHR (\textit{Low Energy High Resolution}) atau LEUR(\textit{Low Energy Ultra-High Resolution}), variasi sudut $360^\circ$ tiap $5^\circ$, jumlah sampel linear per akuisisi 64, ukuran matriks citra $128 \times 128$ atau $256 \times 256$, radius rotasi sekitar 13 cm. Ketebalan slice piksel adalah 8 cm untuk Cold Rod dan 6 sampai 12 mm untuk bola padat. Filter yang digunakan adalah jenis Ramp, frekuensi cut-off lebih besar atau sama dengan 1.4 putaran/cm, kolimator yang digunakan adalah \textit{High Res}, dan window energi sekitar 15-20\%. Langkah-langkah dalam melakukan eksperimen adalah sebagai berikut:

\begin{itemize}
\item Kalibari SPECT \\
Berdasarkan AAPM No. 6, 22, 52, kalibrasi SPECT dilakukan dengan melakukan pengukuran kamera gamma dengan phantom bar. Tujuan dari kalibrasi adalah untuk melihat kualitas kinerja dari kolimator dan detektor pada kamera gamma dengan meninjau hasil resolusi intrinsik, resolusi kolimator, dan resolusi ekstrinsik, serta penempatan posisi titik tengah pada kamera gamma. Hasil pengukuran akan menunjukkan PMT gain, energi, linearitas, dan uniformitas sebagai koreksi dalam pengukuran selanjutnya menggunakan phantom lain.
\item Pengisian dan memposisikan phantom \\
Phantom pertama diisi dengan air hingga merata. Kemudian phantom diisi radionuklida \textsuperscript{99m}Tc dengan menggunakan kapiler khusus yang terhubung pada isi phantom, sehingga \textsuperscript{99m}Tc yang telah diaduk dan diketahui konsentrasi aktivitasnya dapat diinjeksikan kedalam phantom.
\item Pengukuran sensitivitas \\
Untuk pengukuran sensitivitas dilakukan dengan phantom tanpa sisipan. Kemudian melakukan pengukuran sensitivitas sumber titik pada phantom yang terisi campuran air dan \textsuperscript{99m}Tc dengan dual kepala sistem SPECT dengan diposisikan berdiri.
\item Pengukuran \textit{line spread} \\
Phantom diisi dengan sisipan \textit{cold rods} dan dimasukkan air dan radionuklida \textsuperscript{99m}Tc. Untuk melihat line spread, phantom diletakkan dengan posisi terbaring pada bed sehingga sumber garis akan parallel dengan sumbu rotasi sistem SPECT.
\item Perhitungan statistikal noise \\
Setelah melakukan rekonstruksi citra dengan sistem SPECT, hasil data piksel yang ada di komputer kemudian dilakukan perhitungan noise secara statistik. Estimasi noise diperoleh dengan menghitung densitas cacahan rata-rata sampel ($Y_{avg}$) dengan persamaan:

\begin{eqnarray}
Y_{avg} = \frac{1}{N} \sum_{i=1}^{N} y_i \label{eq9}
\end{eqnarray}

Dimana \textit{N} adalah jumlah piksel pada ROI (\textit{Region of Interest}) dekat titik tengah citra rekonstruksi bagian silinder uniform phantom SPECT, dan $y_i$ adalah jumlah cacahan  per piksel ke-\textit{i}. Dari hasil tersebut dihitung standar deviasi:

\begin{eqnarray}
S_{dev} = \frac{1}{\sqrt{N-1}}\sqrt{\sum_{i=1}^{N}(y_i - Y_{avg})^2} \label{eq10}
\end{eqnarray}

Dan rms sampel noise dalam persen diperoleh:

\begin{eqnarray}
\%rms = (S_dev/Y_{avg})\ast 100 \label{eq11}
\end{eqnarray}

\item Kontras citra \\
Rekonstruksi citra dari phantom yang menggunakan sisipan bola padat atau \textit{cold rods} akan diperoleh pengukuran kuantitatif kontras ($C_{image}$) untuk sisipan bola:

\begin{eqnarray}
C_{image} = \left|\frac{(Counts\cdot {Pixel}^{-1})_{sphere}-(Counts\cdot {Pixel}^{-1})_{background}}{(Counts\cdot {Pixel}^{-1})_{background}}\right| \label{eq12}
\end{eqnarray}

Untuk \textit{cold rods}:

\begin{eqnarray}
C_{image} = \left|\frac{(Counts\cdot {Pixel}^{-1})_{cold rods}-(Counts\cdot {Pixel}^{-1})_{background}}{(Counts\cdot {Pixel}^{-1})_{background}}\right| \label{eq13}
\end{eqnarray}

\item Pendeteksian artefak \\
Investigasi artefak dilakukan pengukuran inhomogenitas dengan phantom tanpa sisipan. Teknik scan berdasarkan teknik pencacahan dari kamera gamma dan ditampilkan pada SPECT.
\end{itemize}

Untuk metode simulasi digunakan perangkat lunak GATE dengan langkah awal mendesain sistem scan SPECT dan phantom Jazszcak, lalu membuat protokol optimisasi, melakukan tes algoritma, memperoleh data kuantifikasi, melakukan koreksi hamburan, menganalisis data, dan terakhir rekonstruksi citra.

\subsection{GATE}
Salah satu code simulasi Monte Carlo yang digunakan adalah GATE. Code GATE merupakan salah satu dari program geant4 berbasis bahasa pemrograman C++ pada OS linux. Fitur dari simulasi GATE ini terdiri dari deskripsi akurat fenomena yang bergantung waktu seperti gerakan detektor, pernafasan pasien, gerakan kadiak, kinetika radiotracer, dsb. Pada aplikasi GATE terdapat perhitungan sistem matriks yang menggunakan rekonstruksi dan produksi data untuk evaluasi koreksi algoritma. Penggunaan GATE sebagai contoh dalam menentukan dosimetri, simulasi studi klinis dan fokus optimisasi algortima.

Fitur-fitur yang disediakan GATE untuk melakukan simulasi adalah:

\begin{enumerate}
\item Konstruksi geometri dari modalitas (SPECT atau PET) dalam melakukan simulasi, dan bentuk dari phanto yang akan digunakan.
\item Proses fisik yang terjadi dalam simulasi meliputi interaksi-interaksi yang terjadi pada foton atau elektron.
\item Sumber yang digunakan beserta data spesifikasinya.
\item Pergerakan dari foton dan elektron berupa jenis, kecepatan, atau komponen lainnya dalam melihat jalur interaksi sehingga dapat ditinjau \textit{cross-section} dan probabilitasnya. Durasi waktu akuisisi pengambilan data
\end{enumerate}

Output data dari GATE dalam bentuk ROOT, file ASCII, file binary, dan format data spesifik scanner yang disimulasikan.

\subsection{Model Simulasi GATE}
Data kalibrasi gamma kamera yang diperoleh (resolusi sistem, energi, uniformitas, linearitas) akan digunakan dan dimasukkan dalam GATE. Spesifikasi dari modalitas pesawat SPECT yang digunakan pada pengukuran manual juga dimasukkan sebagai parameter dalam mendesain SPECT yang ada di GATE. Karena menggunakan phantom Jaszczak maka dimasukkan pula parameter dari phantom ke dalam GATE. Simulasi akan dilakukan dengan membuat desain eksperimen dalam keadaan ideal dan sesuai dengan eksperimen manual. Total partikel yang akan disimulasikan sekitar 10 juta partikel. Beberapa hal yang perlu diperhatikan dalam modelling bergantung waktu diantaranya:

\begin{enumerate}
\item Perubahan distribusi tracer terhadap waktu.
\item Gerakan detektor dalam akuisisi.
\item Peluruhan radioaktif.
\item \textit{Dead time} pada detektor.
\end{enumerate}

Prinsip dasar dari modelling waktu di GATE adalah menyesuaikan pemodelan peluruhan radioaktif berdasarkan skema peluruhan yang akan terjadi. Kemudian mendefinisikan variabel waktu ketika simulasi eksperimen sehingga data akan berubah berdasarkan perubahan waktu.

\begin{figure}[h]
\includegraphics[scale=0.5]{Gambar/simulation1.png}
\caption{Contoh model simulasi desain SPECT pada GATE}
\label{simulation1}
\end{figure}

\begin{figure}[h]
\includegraphics[scale=0.6]{Gambar/simulation2.png}
\caption{Modelling konstruksi geometri pada GATE dengan command script}
\label{simulation2}
\end{figure}

\begin{figure}[h]
\includegraphics[scale=0.7]{Gambar/simulation3.png}
\caption{Modelling peluruhan sumber radioaktif}
\label{simulation3}
\end{figure}

\begin{figure}[h]
\includegraphics[scale=0.7]{Gambar/simulation4.png}
\caption{Data output simulasi GATE}
\label{simulation4}
\end{figure}

Kemudian modelling respon dari sistem elektronik yang digunakan dalam simulasi pencitraan. Komponen-komponen dan cara kerja dari beberapa rangkaian digital perlu diketahui dalam melakukan processing data simulasi. Pada bagian ini diusahakan mereproduksi kurva \textit{count rate} secara akurat.

Untuk memudahkan desain detektor simulasi dapat dipecah menjadi dua bagian, yaitu transportasi foton pada objek dan transportasi foton pada detektor. Output dari transportasi foton pada objek direkam pada ruang fase. Teknik ini berguna dalam mengetes detektor yang didesain tanpa melakukan pengulangan.

Algoritma yang digunakan dalam simulasi mengikuti gambar \ref{mc}. Kemudian dalam pendeteksian artefak, jenis ring artefak yang akan ditinjau. Pada simulasi tembaga (Cu) atau timbal (Pb) ditempel pada kolimator sehingga dari hal tersebut akan melihat rekonstruksi dari ring artefak sehingga analisis dari artefak yang dihasilkan akan dibandingkan dengan pengukuran manual yang tidak menggunakan Cu dan Pb. Kemungkinan jika pada eksperimen manual tidak muncul artefak, maka hanya dianalisis artefak yang muncul disimulasikan. Optimalisasi protokol dan algoritma yang digunakan mengikuti gambar (\ref{mc}).

\section{Diagram Alir Penelitian}
\begin{figure}[h]
\includegraphics[scale=0.5]{Gambar/diagpen.png}
\caption{Diagram Alir Penelitian}
\label{diagpen}
\end{figure}
